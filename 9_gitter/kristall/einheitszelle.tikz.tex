
  
\begin{tikzpicture}[style={line width=1pt, shorten >=1mm }]

%\useasboundingbox (-1.3,-1.2) rectangle (11.2,4.7);
%	\draw (-1.5,-2.5) rectangle (3.5,7.5);
  \tikzmath{\x1 = 0;  \dx = 1; \x2 = 1.5 ; \w = 0.7; \bs = 0.8;  \ylu = -1.7; \ylo = -1; \arrowoffs = 0.12; \anh = 0.01;}

    \clip (-2.2cm,-3cm) rectangle (3cm,3cm); 
  \pgftransformcm{1}{0.3}{0.8}{1}{\pgfpoint{0cm}{0cm}}
    
  \fill[gray]  (-3,2) -- ++ (0,1) -- ++ (1,0) -- ++ (0,-1) -- cycle;
  \draw[->] (-3,2) -- ++ (0,1);
  \draw[->] (-3,2) -- ++ (1,0);
  
        \fill[gray] (-2,0) -- ++ (1,-1)  -- ++ (0,1) -- ++ (-1,+1) -- cycle;
      \draw[->] (-2,0) -- ++ (0,1);
   \draw[->] (-2,0) -- ++ (1,-1);
   
     \fill[gray] (2,-3) -- ++ (-1,2) -- ++ (-2,3) -- ++ (1,-2) -- cycle;
    \draw[->] (2,-3) -- ++ (-2,3);
\draw[->] (2,-3) -- ++ (-1,2);
  
  
   \fill[black!20!white] (1,0) -- ++ (1,0) -- ++ (-1,2) -- ++ (-1,0) -- cycle;
      \draw[->] (1,0) -- ++ (-1,2);
      \draw[->] (1,0) -- ++ (1,0);
      
       \fill[black!20!white]  (5,-4) -- ++ (1,0)  -- ++ (-2,2) -- ++ (-1,0) -- cycle;
   \draw[->] (5,-4) -- ++ (-2,2);
   \draw[->] (5,-4) -- ++ (1,0);
     
         \foreach \x in {-7,-6,...,7}{%  
      \foreach \y in {-7,-6,...,7}{% 
           \node[draw,circle,inner sep=1pt,fill] at (\x,\y) {};
      }
    }
    
  
  \end{tikzpicture}




