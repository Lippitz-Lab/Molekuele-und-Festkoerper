\renewcommand{\lastmod}{\ \ }
\renewcommand{\chapterauthors}{\ \ }

\chapter*{Vorwort}

Dies ist das Vorlesungsskript meiner Vorlesung 'Molekülphysik und Festkörperphysik I', die ich im zweiten Corona-Semester im Winter 2020/21 gehalten habe. Sie ist eine Kursvorlesung für  Studierende im dritten Jahr des Bachelorstudiums. Bei der Auswahl und Gewichtung der Themen folgt sie sehr stark dem in Bayreuth Üblichen. Ich danke an dieser Stelle insbesondere Jürgen Köhler und Anna Köhler, deren Vorlesungsskripte mir eine große Hilfe waren.


Im Corona-Semester habe ich dieses Skript (nach und nach) an die Studierenden  ausgeteilt. Zu jedem Kapitel gibt es ingesamt circa eine Stunde 'Vorlesung' auf Video, in der ich mündlich durch den Text führe und dabei an den Rand kritzle.
Ich habe den Eindruck, dass es mir beim Sprechen leichter fällt, die Dinge in einen Zusammenhang zu bringen als beim Schreiben, da ich mich traue, schlampiger zu sein. Zur Vorbereitung gab es dann noch ein online multiple-choice Quiz, sowie die Möglichkeit, jederzeit anonym Fragen zu stellen.\sidenote{\href{http://frag.jetzt}{ frag.jetzt}}  In Live-Videokonferenzen besprachen wir offene Fragen und diskutierten Aufgaben ähnlich zu Erik Mazurs 
ConcepTests.\sidenote{\href{https://mazur.harvard.edu/research-areas/peer-instruction}{mazur.harvard.edu}}  Schließlich gab es die in der Physik üblichen Übungszettel und Kleingruppen-Übungen.



Dieses Skript ist 'work in progress', und wahrscheinlich nie wirklich fertig.  Ich danke allen Studierenden des 2020er Jahrgangs, die den Text und die Gleichungen aufmerksam gelesen haben, wodurch wir viele Fehler korrigieren konnten. Trotzdem wird es noch welche geben. Wenn Sie Fehler finden, sagen Sie es mir bitte. 
Die aktuellste Version des Vorlesungsskripts finden Sie auf meiner 
Website.\sidenote{\href{http://www.ep3.uni-bayreuth.de/lecturenotes}{ep3.uni-bayreuth.de}} Dort verlinke ich auch die Videos. Ich habe alles unter eine CC-BY-SA-Lizenz gestellt (siehe Fußzeile). In meinen Worten: Sie können damit machen, was Sie wollen. Wenn Sie Ihre Arbeit der Öffentlichkeit zur Verfügung stellen, erwähnen Sie mich und verwenden Sie eine ähnliche Lizenz. 


Der Text wurde mit der LaTeX-Klasse 'tufte-book' von Bil Kleb, Bill Wood und Kevin Godby\sidenote{\href{https://tufte-latex.github.io/tufte-latex/}{tufte-latex}} gesetzt, die sich der Arbeit von Edward Tufte\sidenote{\href{https://www.edwardtufte.com/}{edwardtufte.com}} annähert. Ich habe viele der Modifikationen angewandt, die von Dirk Eddelbüttel im R-Paket 'tint' eingeführt wurden\sidenote{\href{https://dirk.eddelbuettel.com/code/tint.html}{tint: tint is not Tufte}}. Die Quelle ist vorerst LaTeX, nicht Markdown.




\vspace{2\baselineskip}

Markus Lippitz \\ Bayreuth, 23. Januar 2021

 
 



