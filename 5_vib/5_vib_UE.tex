\renewcommand{\lastmod}{\today}

\chapter{Vibrationsspektroskopie}



\section{Anharmonischer Oszillator}
% ML 14


Die Schwingungsbewegung der Atomkerne in Molekülen kann in vielen Fällen durch harmonische Oszillatoren modelliert werden. Bei höheren Anregungsenergien wird diese Näherung zunehmend schlechter, weil das Potential von der Parabel abweicht. Eine Alternative zur harmonischen Näherung ist das Morsepotential,

$$ U(R) = U_0 \cdot \left(1 - e^{- \beta (R - R_0)}\right)^2\quad. $$

Die Wahl der Parameter $U_0$, $R_0$ und $\beta$ muss an das jeweilige Molekül angepasst werden.


\begin{itemize}

	\item[\textbf{(a)}] Formen Sie den Ausdruck für $U(R)$ um und zeigen Sie, dass das Morsepotential als Summe eines kurzreichweitigen repulsiven und eines langreichweitigen attraktiven Anteils dargestellt werden kann.

	\item[\textbf{(b)}] Zeigen Sie, dass $R_0$ dem Gleichgewichtszustand entspricht. Welchen Wert nimmt das Potential für den Grenzfall $R \rightarrow \infty$ an? Welche Rolle spielen die Parameter $\beta$ und $U_0$?

	\item[\textbf{(c)}] Zeigen Sie, dass das Morsepotential um den Gleichgewichtsabstand $R \approx R_0$ harmonisch genähert werden kann. Wie lautet die Kraftkonstante $K = m\omega^2$ des genäherten Potentials? 
	
	\item[\textbf{(d)}] Die Eigenenergien im Morsepotential sind gegeben durch 
	
		$$ E_n = \hbar \omega_0 \cdot \left(n + \frac{1}{2}\right) - \frac{\hbar^2 \omega_0^2}{4 U_0} \cdot \left(n + \frac{1}{2}\right)^2 $$
		
		Wieviele gebundene Zustände gibt es? Bestimmen Sie die Parameter $\beta$ und $U_0$ für Wasserstoff (Kraftkonstante $K = 573 N/m$, Dissoziationsenergie $U_{d} = 4.52 eV$ und Gleichgewichtsabstand $R_e = 74 pm$).
	
\end{itemize}



\section{Schwingungsspektroskopie}
\vspace{0.7cm}%ML14
%\begin{itemize}

%	\item[\textbf{(a)}] 
Welche der folgenden Moleküle zeigen IR-aktive Schwingungsmoden (jeweils mit Begründung): N$_2$, C$_2$H$_4$, CH$_3$OH, HD, CCl$_4$, CS$_2$, SO$_2$, NH$_3$, BeCl$_2$ (linear), CH$_3$COCH$_3$?

%\end{itemize}




\section{Rotations- und Schwingungsspektrum von CO}
ML14

Am Beispiel von Kohlenmonoxid wollen wir genauer auf die Struktur der Schwingungsspektren von zweiatomigen Molekülen eingehen. Das CO-Molekül hat eine Bindungslänge von $R_0=1.13 \cdot 10^{-10} m$. Betrachten Sie das unten abgebildete Absorptionsspektrum von CO-Gas bei Raumtemperatur (Quelle: Haken-Wolf). Gezeigt sind Übergänge zwischen verschiedenen Rotationszuständen um den Vibrationsübergang $\nu=0 \rightarrow \nu=1$ innerhalb des elektronischen Grundzustandes. Die fehlende Linie bei $\nu_0=2143 cm^{-1}$ entspricht dem Rotationsübergang $k=0 \rightarrow k=0$. \\
\vspace{3mm}


  \hspace*{5mm}
      \includegraphics[width=0.9\textwidth]{\currfiledir co_spektrum.png}

\vspace*{1\baselineskip}

\begin{itemize}
 											
	\item[\textbf{(a)}] Bestimmen Sie aus der Frequenz der fehlenden Linie die Federkonstante $D$ des Moleküls. 

	\item[\textbf{(b)}] Ordnen Sie ausgehend von der Auswahlregel für Rotationsübergänge die Linien des R- und P-Zweigs den erlaubten Übergängen zu. Wie kommen die unterschiedlichen Intensitäten der Linien zustande? Welcher Rotationszustand ist bei Raumtemperatur am stärksten besetzt?
		
	\item[\textbf{(c)}] Vergleichen Sie die Energieskalen für Rotation und Schwingung beim CO-Molekül (mit Rechnung). In welchem Wellenlängenbereich liegen die jeweiligen Übergänge?

\end{itemize}



\section{Normalschwingungen von CO$_2$}
%ML14

\begin{itemize}

\item[\textbf{(a)}] Bestimmen Sie die Eigenschwingungen des linearen Moleküls CO$_2$ in einer einfachen Näherung. Nehmen Sie hierfür an, dass die Kerne bei Verschiebung entlang der Molekülachse mit der Federkonstante K$_1$ und bei Verschiebung senkrecht zur Molekülachse mit der Federkonstante K$_2$ aneinander koppeln. Stellen Sie die entsprechenden Bewegungsgleichung auf und bestimmen Sie aus dem Gleichungssystem die Eigenmoden und Eigenwerte in Abhängigkeit von den Massen und Federkonstanten. Zeichnen Sie schematisch die Eigenmoden in ein Diagramm ein. Wie verändert sich dieses, wenn das Molekül asymmetrisch macht, also beispielsweise unterschiedliche Massen oder Federkonstanten für die beiden Sauerstoffatome wählt?

\item[\textbf{(b)}] Wie läßt sich diese Näherung auf Wassermoleküle anwenden? (keine Rechnung)

\item[\textbf{(c)}] Wie würden die Bewegungsgleichungen lauten, wenn man Verschiebungen entlang und quer zur Molekülachse nicht separieren könnte? 


\end{itemize}



\section{Schwingungsspektren}
%ak17

a) Welche der folgenden Moleküle zeigen im IR ein Schwingungsabsorptionsspektrum: H2, HCl, CO ,H O,CH Cl,C H ,N ,N-?
Bei einer Messung an 14N16O findet man für die Frequenzen der ersten beiden Schwingungsübergänge 1876, 06 cm-1 ("Grundschwingung") und 3724, 02 cm-1 ("erster Oberton"). Behandeln Sie N O im Rahmen des anharmonischen Oszillators und bestimmen Sie

b) die Schwingungskonstante nu e

c) die Anharmonizitätskonstante x e

d) die Nullpunktsenergie

Die maximal mögliche Schwingungsenergie des anharmonischen Oszillators kann analytisch berechnet werden, indem man im Ausdruck für die Schwingungsenergie die Quantenzahl als kontinuierliche Variable betrachtet.

e) Bestimmen Sie damit die Dissoziationsenergie D0 und vergleichen Sie diese mit dem gemessenen Wert von 5, 91 eV . Wodurch kommt die Abweichung zustande?


\section{Heiße Banden}
% AK 17
Das Iodmolekül I2 hat die Schwingungskonstante nu e = 215 cm-1 und die Anharmonizitätskonstante xe = 0.003 (anharmonischer Oszillator). Welches Intensitätsverhältnis zwischen der heißen Bande (nu=1 rightarrow nu=2)und der Grundschwingungsbande (nu=0 rightarrow nu=1) erhält man bei T =300K?


\section{Schwingungsspektrum von Toluol}
% AK17

Welchen charakteristischen Schwingungen von Molekülgruppen entsprechen die in untenstehendem IR-Spektrum von Toluol gekennzeichneten Peaks bzw. Wellenzahlbereiche (A-D)?

(Quelle: http://webbook.nist.gov/chemistry)


\section{Rotations-Schwingungsspektrum von 1H35Cl}
%AK17
 Die experimentell für das 1H35Cl-Molekül gefundenen Strukturdaten sind:
Bindungslänge Re: 127, 5 pm
Kraftkonstante der Bindung k : 516, 3 Nm
Atommassen m: 1H : 1, 673 · 10-27 kg, 35Cl : 58, 066 · 10-27 kg
Ermitteln Sie daraus unter Annahme eines harmonischen Oszillators bzw. eines starren Rotators (Vernachlässigung der Kopplung zwischen Rotation und Schwingung)

a) die Nullpunktsenergie E0 für Schwingungen,

b) die Rotationskonstante B,

c) die spektrale Lage (in cm-1) der jeweils innersten drei Linien von P- und R-Zweig. Vernachlässigen Sie dabei die Effekte der Zentrifugal- und der Schwingungsdehnung.

d) Wodurch kommen die Unterschiede zwischen dem in c) berechneten und einem tatsächlich gemessenen Spektrum zustande?
Berücksichtigen Sie im Folgenden nun, dass die Rotationskonstante B aufgrund der Schwingungs- dehnung von der Schwingungsquantenzahl v abhängt, d.h. B = B(v)(= Bv). Vernachlässigen Sie
n

e) Ermitteln Sie für die Grundschwingungsbande von 1H35Cl die Wellenzahl nu R(JK) der Bandkante des R-Zweiges. Dabei ist JK derjenige der Werte von J, bei dem mit wachsendem J die Rotationslinien anfangen, auf der Wellenzahlskala umzukehren.
Hinweis: Der Abstand der (J = 0)-Zustände der Schwingungsniveaus nu = 0 und nu = 1 von 1H35Cl
-1
beträgt nu(1,0) = 2885,9cm . Die Rotationskonstanten für nu = 0 und nu = 1 betragen B0 =
 jedoch zur Vereinfachung die Dehnungskorrekturen höherer Ordnung als n = 1 von [J · (J + 1)] .
-1




