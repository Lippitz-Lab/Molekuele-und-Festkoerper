\renewcommand{\lastmod}{\today}

\chapter{Dielektrische Eigenschaften}



\section{Kramers-Kronig-Relation}

%ML 14

Sie wollen durch eine Messung die Frequenzabhängigkeit der dielektrischen Konstante $\epsilon = \epsilon' + i\cdot \epsilon''$ einer Substanz bestimmen. Da Ihnen experimentell bedingt aber nur der Imaginärteil zugänglich ist, planen Sie, den Realteil mit Hilfe der Kramers-Kronig-Beziehung aus den Messdaten zu bestimmen. 

Es gilt also

$$ \Im[\epsilon(\omega)] = - \frac{2}{\pi} CH \int\limits_{0}^{\infty} \frac{\omega \cdot \Re[\epsilon(\Omega)]}{\Omega^2 - \omega^2} d\Omega$$
$$ \Re[\epsilon(\omega)] =   \frac{2}{\pi} CH \int\limits_{0}^{\infty} \frac{\Omega \cdot \Im[\epsilon(\Omega)]}{\Omega^2 - \omega^2} d\Omega\quad,$$

wobei CH den Cauchy-Hauptwert kennzeichnet. 

Um die Anforderungen an Ihr Experiment abzuschätzen untersuchen Sie die praktische Anwendbarkeit der Kramers-Kronig-Relation zur Bestimmung des Realteils des Brechungsindex. Wichtige Parameter sind dabei die Breite des gemessenen Frequenzintervalls, die Anzahl der Messpunkte über dieses Intervall und das Signal-zu-Rausch-Verhältnis. Um diese Größen zu untersuchen müssen Sie sich zunächst künstliche Messdaten mit Hilfe eines analytischen Models erzeugen.

\begin{itemize}

	\item[\textbf{(a)}] Sie wissen, dass die optische Antwort der untersuchten Substanz näherungsweise durch einen gedämpften, harmonischen Oszillator modeliert werden kann. Der Brechungsindex ist dann analytisch gegeben durch 
	
		$$\epsilon' - i\cdot \epsilon'' = \frac{e \cdot m \cdot (\omega^2 - \omega_0^2)}{m^2 \cdot  (\omega^2 - \omega_0^2)^2 + \gamma^2 \omega^2} - i \cdot \frac{e \cdot \gamma \cdot \omega}{m^2 \cdot  (\omega^2 - \omega_0^2)^2 + \gamma^2 \omega^2}\quad,$$

	wobei $\gamma$ die Dämpfung, $\omega_0$ die Resonanzkreisfrequenz und m die Elektronmasse ist. Benutzen Sie diese Gleichung um einen Satz künstlicher Messdaten für $\epsilon''$ zu erzeugen, das heißt einen diskreten Satz von Frequenzen und die dazugehörigen Werte von $\epsilon''$. Machen Sie sich insbesondere Gedanken über die Frequenzbandbreite und die Frequenzschrittweite im Verhältnis zur Breite der Resonanzkurve (gegeben durch $\gamma$). Für die Parameter müssen keine realistischen Werte angenommen werden.
	
	\item[\textbf{(b)}] Benutzen Sie nun die Kramers-Kronig-Beziehung um aus den künstlichen Messdaten für $\epsilon''$ den Realteil von $\epsilon$ zu berechnen. Da Sie diskrete Werte für $\omega$ haben, können Sie das Integral durch eine Summe ersetzen. Welchen Einfluss hat die Bandbreite und die Schrittweite der Frequenzachse auf das Ergebnis? Vergleichen Sie Ihr Ergebnis mit der analytischen Lösung des Oszillatormodels. 
	
	\item[\textbf{(c)}] Untersuchen Sie den Einfluss von Rauschen auf Ihre Messung. Simulieren Sie verrauschte Messdaten, indem Sie auf Ihr Signal normalverteilte Zufallswerte aufaddieren. Variieren Sie die Standardabweichung der Normalverteilung, um unterschiedliche Signal-Rausch-Verhältnisse zu erzeugen. 
	
\end{itemize}



\section{Permanenter elektrischer Dipol}

%ML 14

Moleküle mit permanenten elektrischem Dipolmoment neigen in einem äußeren elektrischen Feld dazu, sich entlang der Feldrichtung zu orientieren. Dem wirkt eine thermische Unordnung entgegen. 

\begin{itemize}

	\item[\textbf{(a)}] Gemäß den Regeln der statistischen Mechanik lässt sich die Wahrscheinlichkeit dafür berechnen, dass sich ein molekularer Dipol p im Feld E vom Winkel $\theta$ zwischen Dipol- und Feldrichtung nach $\theta + \delta \theta$ verdreht. Versuchen Sie daraus, den thermischen Mittelwert $\overline{\cos \theta} = L(\frac{p\cdot E}{k \cdot T})$ abzuleiten. Dabei ist $L(x)$ die \textit{Langevin-Funktion}

	$$ L(x) = \coth(x) - \frac{1}{x}\quad.$$

	\item[\textbf{(b)}] Wie groß muss die Feldstärke sein, um ein Wassermolekül ($p = 6.17\cdot 10^{-30} A\cdot s \cdot cm$) bei Raumtemperatur exakt in Feldrichtung zu orientieren?
	
	\item[\textbf{(c)}] Wasser sei in sehr niedriger Konzentration in (unpolarem) n-Hexan gelöst. Welche Feldstärke benötigt man um bei Raumtemperatur $50\%$ der theoretisch möglichen Orientierungspolarisation zu erhalten?
	
	\item[\textbf{(d)}] Welcher Polarisierungsgrad lässt sich bei einer realistischen Feldstärke von $E = 10^5 V/cm$ bei Kühlung auf die Temperatur flüssigen Heliums (4.2 Kelvin) erreichen?

\end{itemize}



\section{Harmonischer Oszillator}

% ML 14

Wie in der Vorlesung besprochen wurde, kann die Frequenzabhängigkeit der Dielektrizitätskonstanten $\varepsilon$ bzw. des Brechungsindex $n$ eines Moleküls in einem einfachen Modell durch einen gedämpften harmonischen Oszillator angenähert werden. Es gilt folgende Bewegungsgleichung:  
		
		$$m \ddot{x} + \gamma \dot{x} + m \omega_0^2 = e E_0 e^{i\omega t}$$
		
wobei $x$ die Auslenkung, $m$ die Masse, $\omega_0$ die Eigenfrequenz und $\gamma$ die Dämpfungskonstante des Oszillators ist. Das elektrische Feld des anregenden Lichts hat die Amplitude $E_0$ und schwingt mit der Frequenz $\omega$. $e$ ist die Elementarladung.

\vspace*{1\baselineskip}

\begin{itemize}

	\item[\textbf{(a)}] Betrachten Sie die stationäre Lösung $x(t) = X e^{i\omega t}$ und plotten Sie die Frequenzabhängigkeit von reeller Amplitude und Phase der komplexen Amplitude $X$ für verschiedene Werte von $\gamma/\omega_0$. Skizzieren Sie außerdem Real- und Imaginärteil von $X$.
	
	\item[\textbf{(b)}] Berechnen Sie $\varepsilon$ und $n$ unter der Annahme, dass für das Dipolmoment des Moleküls $p=ex$ gilt. Was ist die physikalische Bedeutung des Real- und Imaginärteils von $n$? Erklären Sie anschaulich den Frequenzverlauf der beiden Anteile. Leiten Sie Ausdrücke für die Position des Absorptionsmaximums und für dessen Linienbreite her.
	
\end{itemize}


\section{Orientierungs- und Verschiebungspolarisation}
%AK17

a) Berechnen Sie die makroskopische Polarisation in einem Ensemble permanenter elektrischer
 Dipolmomente p (z.B. in einer Flüssigkeit), welche sich in einem elektrischen Feld E befinden.
   Nehmen Sie für die Wahrscheinlichkeitsverteilung eine Boltzmannverteilung an. Das Ergebnis enthält die Langevin-Funktion L(x) mit x = p·E , wobei p und E den Betrag des elektrischen
kB ·T
Dipolmomentes bzw. des elektrischen Feldes, kB die Boltzmann-Konstante und T die absolute
Temnperatur bezeichnen.
b) Entwickeln Sie das Ergebnis von a) für hohe Temperaturen (Curie-Gesetz).
c) Schätzen Sie für das Molekül HF ab, ab welcher elektrischen Feldstärke die Hochtemperaturnäherung aus b) zusammenbricht. Diskutieren sie das Ergebnis hinsichtlich der Stabilität des Moleküls bei dieser Feldstärke (Hinweis: Vergleichen Sie diese Feldstärke mit der eines (klassischen) Dipols mit der Bindungslänge von HF, s.u.)
d) Schätzen Sie für HF die Frequenzen ab, ab denen der Beitrag zur Orientierungs- bzw Verschiebungspolarisation verschwindet. Rechnen Sie diese Frequenzen auch in Wellenzahlen um. In welchem Spektralbereich liegen diese Frequenzen?
Für HF gilt:
Bindungslänge: R0 = 9, 5 · 10 11 m
permantes Dipolmoment: pp = 6 · 10 30 Cm


\section{Rotationsfrequenz eines Moleküls}
% JK19

Die Umorientierung ganzer Moleküle (Orientierungspolarisation) kann Frequenzen des
Feldes folgen, die ungefähr der inversen Rotationszeit der Moleküle entsprechen. Die
|L| Größenordnung dieser Frequenz kann abgeschätzt werden mit omega rot = 2pi nu = theta .
Für das Trägheitsmoment gilt theta = mRR02, wobei mR die reduzierte Masse ist und R0 die Bindungslänge.
Schätzen Sie die Rotationsfrequenz für kleine Moleküle (z.B. HCl) ab.


\section{Schwingungsfrequenz eines Moleküls}
%JK19
Die Verschiebung von Ladungen im Molekül kann zwei Ursachen haben:
• Die Verschiebung der Kerne relativ zueinander
• Die Verschiebung der Elektronenwolke relativ zu den Kernen
a) Die Frequenzen der Verschiebung der Kerne relativ zueinander liegen in der Größenordnung der Molekülschwingungsfrequenzen. Die Schwingungsfrequenzen können über die Rückstellkraft FR = - k(R - R0) abgeschätzt werden. Schätzen Sie die Schwingungsfrequenz eines Moleküls (z.B. HCl) ab unter der Annahme, dass die Bindungskraft durch eine reine Coulombkraft zustande kommt.
b) Da die Elektronen eine viel geringere Masse als die Kerne besitzen, können sie höheren Frequenzen noch folgen. Die Resonanzen der Elektronen sind Valenz- elektronenübergänge. Deren Frequenzen werden in der Größenordnung der elek- tronischen Übergänge im Atom liegen. Berechnen Sie die Übergangsenergie ato- marer Energieniveaus, um typische Werte für die Frequenz der Verschiebung der Elektronenwolke zu erhalten.