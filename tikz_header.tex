\usepackage{tikz}



\newcommand{\inputtikz}[1]{%
%  \tikzsetnextfilename{#1}%
  \input{#1.tikz}%
}


\usetikzlibrary{math,matrix,fit,positioning}

\usetikzlibrary{calc}
\usetikzlibrary{arrows.meta} %needed tikz library

\usepackage{standalone}
\usepackage{pgfplots}
 \pgfplotsset{compat=newest}
\usepgfplotslibrary{groupplots}

\tikzset{>=latex}

\usepackage{tikzorbital}


\pgfplotsset{
tufte line/.style={
    axis line style={draw opacity=0},
    ytick=\empty,
    axis x line*=bottom,
    x axis line style={
      draw opacity=1,
      gray,
      thick
},
 %   yticklabel=\pgfmathprintnumber{\tick}
  }
  }

\tikzset{
mymat/.style={
    matrix of math nodes,
    left delimiter=|, right delimiter=|,
    align=center,
    column sep=-\pgflinewidth,
}
%,mymats/.style={
%    mymat,
%    nodes={draw,fill=#1}
%} 
 }
 
\newcommand{\myarrow}[5]{\draw[#4](#1.south -| #2)  -- ++(#3 :6mm) node[above,pos=0.55]{$#5$};
} 

\newcommand{\interactLp}[3]{\myarrow{#1-#2-1}{#1.west}{-135}{<-}{#3}} 
\newcommand{\interactLm}[3]{\myarrow{#1-#2-1}{#1.west}{+135}{->}{#3}} 
\newcommand{\interactRp}[3]{\myarrow{#1-#2-2}{#1.east}{ -45}{<-}{#3}} 
\newcommand{\interactRm}[3]{\myarrow{#1-#2-2}{#1.east}{ +45}{->}{#3}}  

\newcommand{\interactout}[2]{\myarrow{#1-1-1}{#1.west}{+135}{->,dashed}{#2}} 


\newcommand{\benzene}[8]{%
\tikzmath{\x1 = #1; \dx1 = 0.5; \dx2 = 0.9; \ps=0.5;}
\tikzmath{\x2 = \x1 + \dx1 ;}
\tikzmath{\x3 = \x2 + \dx2 ;}
\tikzmath{\x4 = \x3 + \dx1 ;}

\tikzmath{\y1 = #2; \dy = 0.5;}
\tikzmath{\y2 = \y1 + \dy ;}
\tikzmath{\y3 = \y2 + \dy ;}

\orbital[pos = {(\x1,\y2)},scale=#3 * \ps]{pz}
\orbital[pos = {(\x2,\y1)},scale=#4 * \ps]{pz}
\orbital[pos = {(\x3,\y1)},scale=#5 * \ps]{pz}
\orbital[pos = {(\x4,\y2)},scale=#6 * \ps]{pz}
\orbital[pos = {(\x3,\y3)},scale=#7 * \ps]{pz}
\orbital[pos = {(\x2,\y3)},scale=#8 * \ps]{pz}

\draw (\x1,\y2) -- (\x2,\y1) -- (\x3,\y1) -- (\x4,\y2) --(\x3,\y3) 
-- (\x2,\y3) -- (\x1,\y2);
}