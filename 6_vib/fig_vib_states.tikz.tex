
  
\begin{tikzpicture}
%\useasboundingbox (-1.3,-1.2) rectangle (11.2,4.7);
%	\draw (-1.5,-2.5) rectangle (3.5,7.5);
  \tikzmath{\x1 = 0;  \dx = 1; \x2 = 1.5 ; \w = 0.7; \bs = 0.8;  \ylu = -1.7; \ylo = -1; \arrowoffs = 0.12; \anh = 0.01;}


 % Zustände
  \foreach \a in {0,...,7}
  {
  		\tikzmath{\b = \w * (\a + 0.5); \xe1 = \x1 + \dx; \banh = \w * (\a + 0.5)  - \w * \anh * (\a + 0.5)^2 ;}
    	\draw[line width=1pt] (\x1,\b) -- (\xe1,\b);    		
    	\draw[line width=1pt] (\x2,\banh) -- ++(\dx,0) node[anchor=west]{\a};
    	\draw[line width=1pt, dotted] (\xe1,\b) -- (\x2,\banh);   	
      }
  
   \node [anchor=west] at (2.5,5.2) {$\nu$};
  \node [anchor=center] at (0.5,5.5) {harm};
 \node [anchor=center] at (2,5.5) {anharm};

  % Übergangs-Pfeile
   \foreach \a in {0,...,6}
   {
  		\tikzmath{\xa1 = \x1 + (\a +1) * \arrowoffs;} 	
  		\tikzmath{\xa2 = \x2 + (\a +1) * \arrowoffs;} 	
  		\tikzmath{\b1 = \w * (\a + 0.5); \banh1 = \w * (\a + 0.5)  - \w * \anh * (\a + 0.5)^2 ;}
     	\tikzmath{\b2 = \w * (\a + 1.5); \banh2 = \w * (\a + 1.5)  - \w * \anh * (\a + 1.5)^2 ;}
  
    	\draw[->] (\xa1,\b1) -- node (transharm\a) {}  (\xa1,\b2) ;
    	\draw[->] (\xa2,\banh1) -- node (transanharm\a) {}  (\xa2,\banh2) ;
  }
  
   \foreach \a in {0,...,2}
   {
  		\tikzmath{\xa2 = 0.45+ \x2 + (\a +1) * \arrowoffs;} 	
  		\tikzmath{\banh1 = \w * (\a + 0.5)  - \w * \anh * (\a + 0.5)^2 ;}
     	\tikzmath{\banh2 = \w * (\a + 2.5)  - \w * \anh * (\a + 2.5)^2 ;}
  
    	\draw[->,red] (\xa2,\banh1) -- node (oberton\a) {}  (\xa2,\banh2) ;
  }
    

   % Energie-Achse
	\draw[->] (0,\ylu) -- (2.7,\ylu) node[anchor = north east,xshift=2mm]{  E. Übergang};  
 	\draw[-] (0,-1.6) -- (0,-1.8) node[below]{0};  

	\draw[->] (-0.4,0) -- (-0.4,5.8) node[anchor = south east, rotate=90 ]{ Energie des Zustands};  
	\draw[-] (-0.3,0) -- (-0.5,0) node[anchor = east]{0};  

  
  % linien
   \foreach \a in {0,...,6}
   {
   		\tikzmath{\b1 = \w * (\a + 0.5); \banh1 = \w * (\a + 0.5)  - \w * \anh * (\a + 0.5)^2 ;}
     	\tikzmath{\b2 = \w * (\a + 1.5); \banh2 = \w * (\a + 1.5)  - \w * \anh * (\a + 1.5)^2 ;}
 
  		\tikzmath{ \xnanh = (\banh2 - \banh1) * 1.5;} 	
    	\draw[line width = 0.1pt] (\xnanh,\ylu) -- (\xnanh,\ylo) node (linie\a) {};
}

   \foreach \a in {0,...,2}
   {
   		\tikzmath{ \banh1 = \w * (\a + 0.5)  - \w * \anh * (\a + 0.5)^2 ;}
     	\tikzmath{\banh2 = \w * (\a + 2.5)  - \w * \anh * (\a + 2.5)^2 ;}
 
  		\tikzmath{ \xnanh = (\banh2 - \banh1) * 1.5;} 	
    	\draw[red] (\xnanh,\ylu) -- (\xnanh,\ylo) node (linieot\a) {};
}




  		\tikzmath{ \xn = \w * 1.5;} 	

    	\draw[line width = 1pt] (\xn,\ylu) -- (\xn,\ylo) node (linie_harm) {};


	\draw [decorate,decoration={brace,amplitude=4pt}, blue] (0.9,0.3) -- node[below=-1pt,left=0pt] (cb) {} (0.1,0.3);

  \draw[->, blue] (cb) .. controls   (1,-0.4) and (\xn,0)   ..  (linie0);
  \draw[->, blue] (transanharm6) .. controls (1,4)  and  (1,0)  ..  (linie6);
   \draw[->, blue] (transanharm0) .. controls (1.0,-0.5) and (\xn,0)   ..  (linie0);
 
   \draw[->, blue] (oberton0) .. controls (2.5,-0.25) and (2*\xn,0)    ..  (linieot0);

\end{tikzpicture}

