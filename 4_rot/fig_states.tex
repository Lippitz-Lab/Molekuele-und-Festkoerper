\documentclass[margin=0mm]{standalone}
\usepackage{tikz}
\usepackage{pgfplots}
 \pgfplotsset{compat=newest}


\usepackage{currfile,hyperxmp}
\usetikzlibrary{math,matrix,fit,positioning}


\begin{document}



  
\begin{tikzpicture}
%\useasboundingbox (-1.3,-1.2) rectangle (11.2,4.7);
	\draw (-1.5,-2.5) rectangle (3.5,7.5);
  \tikzmath{\x1 = 0; \x2 = 1.9; \bs = 0.12;  \ylu = -1.7; \ylo = -1; \arrowoffs = 0.23;}
%  \tikzmath{\x1 = 0; \x2 = 2.4; \bs = 0.16;  \ylu = -1.7; \ylo = -1; \arrowoffs = 0.3;}

 % Zustände
  \foreach \a in {0,...,7}
  		\tikzmath{\b = (\a * (\a + 1) * \bs;}
    	\draw[line width=1pt] (\x1,\b) -- (\x2,\b) node[anchor=west]{\a};
  
   \node [anchor=west] at (\x2,7.1) {J};

  % Übergangs-Pfeile
   \foreach \a in {0,...,6}
   {
  		\tikzmath{\b1 = (\a * (\a + 1)) * \bs;  \b2 = (\a +2)* (\a + 1) * \bs; \xa = \x1 + (\a +1) * \arrowoffs; \xn = \b2 - \b1;} 	
    	\draw[->] (\xa,\b1) -- node (trans\a) {}  (\xa,\b2) ;
  }

   % Energie-Achse
	\draw[->] (0,\ylu) -- (2.2,\ylu) node[anchor = north ]{  E. Übergang};  
 	\draw[-] (0,-1.6) -- (0,-1.8) node[below]{0};  

	\draw[->] (-0.4,0) -- (-0.4,7.0) node[anchor = south east, rotate=90 ]{ Energie des Zustands};  
	\draw[-] (-0.3,0) -- (-0.5,0) node[anchor = east]{0};  

  
  % linien
   \foreach \a in {0,...,6}
   {
  		\tikzmath{\b1 = (\a * (\a + 1)) * \bs;  \b2 = (\a +2)* (\a + 1) * \bs;  \xn = \b2 - \b1;} 	
    	\draw[line width = 1pt] (\xn,\ylu) -- (\xn,\ylo) node (linie\a) {};
}

   \draw[->, blue] (trans6) .. controls (3,6)  and  (3,0)  ..  (linie6);
   \draw[->, blue] (trans0) .. controls (-0.1,-0.5)    ..  (linie0);

\end{tikzpicture}

\end{document}