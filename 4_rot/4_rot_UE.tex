\renewcommand{\lastmod}{\today}

\chapter{Rotationsspektroskopie}


\section{Rotationsspektroskopie}
\vspace{0.7cm}

Vom Molekül $^1$H$^{127}$I wurde folgendes Rotationsspektrum gemessen: \\
\vspace{3mm}


  \hspace*{5mm}
      \includegraphics[width=0.8\textwidth]{\currfiledir rotationsspektrum_hi.pdf}

\vspace*{1\baselineskip}

\begin{itemize}

	\item[\textbf{(a)}] Bestimmen Sie aus diesem Spektrum in der Näherung des starren Rotators Rotationskonstante, Trägheitsmoment und Bindungslänge des Moleküls. Wie oft pro Sekunde rotiert das Molekül bei $J=0$, $J=1$ und $J=5$?
											
	\item[\textbf{(b)}] Berechnen Sie die Intensitäten einiger Rotationslinien unter der Annahme, dass das Spektrum bei einer Temperatur von 680 K gemessen wurde. Vergleichen Sie die berechneten mit den gemessenen Intensitäten.
	
	\item[\textbf{(c)}] In der Näherung des nicht-starren Rotators: Bestimmen Sie die Dehnungskonstante aus dem Rotationsspektrum. Berechnen Sie damit die Federkonstante des HI-Moleküls.

\end{itemize}



\section{Rotationsspektren}
%Ak17

a) An welchen Molekülen könnte man gemäß der Auswahlregeln für Rotationsspektren ein reines Mikrowellen-Rotationsspektrum beobachten: H2, H2O, H2O2, CH4, CH3Cl, CH2Cl2, NH3, NH4Cl, HCl, Br2, HBr, CS2?

b) Bestimmen Sie aus diesem Spektrum im Rahmen der Näherung des starren Rotators
(a) die Rotationskonstante B (b) das Trägheitemoment theta
(c) die Bindungslänge R des Moleküls
(d) die Zahl der Rotationen pro Sekunde für J=0, J=1 und J=5
c) Berechnen sie die Intensitäten der Rotationslinien unter Annahme einer Boltzmann-Verteilung und vergleichen Sie die berechneten mit den gemessenen Intensitäten. Normieren Sie dabei auf den ersten Peak und gehen Sie von einer Temperatur T = 680 K aus. Vernachlässigen sie bei der Berechnung die Abhängigkeit vom Übergangsdipolmoment. Wichtig: Bei dem mit (*) gekennzeichneten Peak handelt es sich um den Übergang von J = 1 auf J = 2.
   1
Besonders bei höheren Werten der Quantenzahl J spielen auch Dehnungseffekte eine Rolle und die Bindung kann nicht mehr als vollkommen starr betrachtet werden (nicht-starrer Rotator).
d) Zeigen Sie zunächst, dass sich die Rotationsenergie im Falle eines nicht-starren Rotators schreiben lässt als:
%Erot = B·h·c·J(J+1) - D·h·c·J2(J+1)2 (1) 
mit der bekannten Rotationskonstane B und der Dehnungskonstante D=3(k: Federkonstante
% 4 π k θ 2 R 02 c
, theta: Trägheitsmoment, R0: Gleichgewichtsabstand der Atome, c: Lichtgeschwindigkeit).
Hinweis: Drücken Sie dazu zunächst mit Hilfe des aus der klassischen Mechanik resultierenden Kräftegleichgewichts bei der Rotation die Rotationsenergie des Systems als Funktion des Drehimpulses L, der reduzierten Masse mr, der Federkonstante k und des Gleichgewichtssabstandes R0 aus. Nehmen Sie dabei an, dass die Abweichung vom Gleichgewichtsabstand delat R = R - R0 sehr klein im Vergleich zu R0 sei. Ersetzen Sie schließlich den Drehimplus L durch den entsprechenden quantenmechanischen Ausdruck.
e) Bestimmen Sie nun die Dehnungskonstante aus dem Rotationsspektrum und berechnen Sie damit die Federkonstante des HI-Moleküls.
