\renewcommand{\chapterauthors}{Markus Lippitz}
\renewcommand{\lastmod}{21. Dezember 2021}


\chapter{Das reziproke Gitter}


\section{Ziele}

\begin{itemize}
\item Sie können das Konzept des reziproken Raums und die Streutheorie benutzen, um Beugungsbilder wie das untenstehende zu erklären und daraus Eigenschaften des Gitters zu bestimmen.

\item Sie können in der Darstellung von Eigenschaften wie beispielsweise der Streudichte zwischen Realraum und reziprokem Raum wechseln, also die Fourier-Transformation intuitiv anwenden.
\end{itemize}


\begin{figure}
\inputtikz{\currfiledir debye_scherrer_film}
  \caption{Skizze des Beugungsbilds in einem Pulver-Diffraktometer nach Debye-Scherrer.}
\end{figure}




\section{Wie miss man das?}

In einem  Pulver-Diffraktometers  nach Debye-Scherrer wird ein Pulver einer kristallinen Substanz von einem monochromatischen Röntgenstrahl durchleuchtet. Die gebeugte Strahlung wird von einem Film detektiert, der in einem Ring um die Probe liegt. Daher sind im Film bei $0^\circ$ und $180^\circ$ Aussparungen, um den Röntgenstrahl durchzulassen. Es finden sich konzentrische elliptische Linien mit jeweils konstantem Ablenkwinkel, der $2\Theta$ genannt wird.

\begin{marginfigure}
\inputtikz{\currfiledir setup}
  \caption{Skizze  eines Pulver-Diffraktometers nach Debye-Scherrer.}
\end{marginfigure}

Dieses Verfahren ist einfacher als das historisch ältere Laue-Verfahren, in dem ein einziger, homogener Kristall als beugendes Element verwendet wird. Im Pulver sind alle Orientierungen  des Kristalls zum einlaufenden Strahl vorhanden. Wie wir unten bei der Ewald-Kugel sehen werden ist es eher unwahrscheinlich, dass die Kombination von Wellenlänge, Einfallsrichtung und Gitter zu konstruktiver Interferenz führt. Im Laue-Verfahren muss der Kristall daher passend orientiert werden und / oder Röntgenstrahlung mit einem breiten Spektrum verwendet werden.

Der in diesem Kapitel vorgestellte Formalismus ist nicht auf die Beugung von Röntgenstrahl beschränkt. Man kann völlig analog auch sichtbares Licht, Elektronen oder Neutronen beugen. Der Zusammenhang zwischen Energie pro Teilchen und Wellenlänge unterscheidet sich. Kurze Wellenlängen sind mit massereichen Teilchen bei niedrigeren Energien zu erreichen.  Auch kann man durch die Wahl des Strahls entscheiden, ob man die Verteilung der Elektronen (Röntgenstrahlen) oder der Kerne (Neutronenstrahlen) untersucht. 


\section{Grundidee der Streutheorie}

Wie wechselwirkt eine Welle mit einer gegebenen Anordnung von Objekten? Diese Frage beantwortet die Streutheorie. In diesem Abschnitt bleiben wir allgemein, spezifizieren weder die Art der Welle (Licht, Elektronen, Neutronen), noch die Art der Objekte (Spalte, Elektronen, Kerne). Die Nomenklatur folgt \cite{Hunklinger2014}.

Eine ebene, hier der Einfachheit halber skalare Welle $A(t)$ also
\begin{equation}
 A(t) = A_0 \, e^{- i \, (\omega_0 t - \mathbf{k}_0 \cdot \mathbf{r})}
\end{equation}
wird durch die Amplitude $A_0$, Kreisfrequenz $\omega_0$ und Wellenvektor $\mathbf{k}_0$ beschrieben. Diese Welle fällt auf das punktförmige Streuzentrum im Ursprung des Koordinatensystems und erzeugt dort eine auslaufende Kugelwelle der Form
\begin{equation}
 A_Z(t) = \frac{\tilde{A}}{R} \,  e^{- i \, (\omega_0 t - k_0 R)}
\end{equation}
wobei  die Kugelwelle nur noch vom Abstand $R = |\mathbf{R}|$ zum Kugelmittelpunkt und dem Betrag $k_0 = | \mathbf{k}_0| $ des Wellenvektors abhängt. Die Amplitude $\tilde{A}$ hängt von der Effizienz des Prozesses ab.

\begin{marginfigure}
\inputtikz{\currfiledir kugelwelle}
\caption{Skizze Streuung am Punkt}
\end{marginfigure}




Eine ausgedehnte Probe besteht damit aus vielen Streuzentren, deren Dichte als \emph{Streudichte} $\rho(\mathbf{r})$ bezeichnet wird. Allerdings muss man die Phasendifferenz der einzelnen Wellen beachten, wenn man über viele Streuzentren integriert. Für zwei Streuzentren, eines im Ursprung, das andere bei $\mathbf{r}$, findet man durch geometrische Überlegungen die Phasendifferenz
%
\begin{marginfigure}
\inputtikz{\currfiledir two-points-interference}
\caption{Skizze zur  Wegdifferenz $dx =  \Delta \phi / | \mathbf{k}|=  ( \mathbf{k} - \mathbf{k}_0 ) \cdot \mathbf{r} / |\mathbf{k}| $ bei zwei Streuzentren.}
\end{marginfigure}
%
\begin{equation}
\Delta \phi = ( \mathbf{k} - \mathbf{k}_0 ) \cdot \mathbf{r} \quad ,
\end{equation}
wenn $\mathbf{k}$ der Wellenvektor der auslaufenden gestreuten ebenen Welle ist.\sidenote{Die Kugelwelle von oben lässt sich als Integral über ebene Wellen in alle Raumrichtungen schreiben.} Nun machen wir die Annahme, dass die Probe klein ist gegenüber $R$. Wir interessieren uns für die gestreute Welle also erst in einem Abstand, der so groß ist, dass unabhängig von Ort des Streuzentrums alle Teilwellen mit dem Wellenvektor $\mathbf{k}$ am gleichen Ort ankommen. Auch nehmen wir wie quasi immer in der Streutheorie an, dass jede Welle nur einmal gestreut wird. Dies ist die  Born'schen Näherung. Damit wird die gestreute Welle
\begin{align}
 A_S(t) = & \frac{\tilde{A}}{R} \,   e^{- i \, (\omega_0 t - k_0 R)} \,
  \int_{V_\text{Probe}} \, \rho( \mathbf{r}) \,   e^{- i \,  ( \mathbf{k} - \mathbf{k}_0 ) \cdot \mathbf{r}} \, dV \\
  = & \frac{\tilde{A}}{R} \,   e^{- i \, (\omega_0 t - k_0 R)} \, \mathcal{A}(\mathbf{K}) 
\end{align}
mit der \emph{Streuamplitude} 
\begin{equation}
\mathcal{A}(\mathbf{K}) =  \int_{V_\text{Probe}} \, \rho( \mathbf{r}) \,   e^{- i \,  ( \mathbf{k} - \mathbf{k}_0 ) \cdot \mathbf{r}} \, dV  = \mathcal{FT} \left\{ \rho( \mathbf{r})  \right\} \label{eq:rezi_streuamplitude}
\end{equation}
und dem \emph{Streuvektor} $\mathbf{K} = \mathbf{k} - \mathbf{k}_0$. In Streuexperimenten misst man also die Fourier-Transformierte der Streudichte. Dies gilt sowohl für die Beugung am Doppelspalt wie für die Beugung von Röntgenstrahlen an der Elektronenverteilung in einem Kristall. 

\section{Das reziproke Gitter}
Wenn die Elektronendichte im Kristall schon als Streudichte via Fourier-Transformation das Beugungsbild als Streuamplitude ergibt, dann kann man sich auch gleich die Fourier-Transformierte beispielsweise der Elektronendichte anschauen.

Die Streudichte $\rho(\mathbf{r})$ ist wie alle Eigenschaften eines Kristalls gitter-periodisch, also 
\begin{equation}
  \rho(\mathbf{r}) = \rho(\mathbf{r} + \mathbf{T}) = \rho(\mathbf{r} + n_1 \mathbf{a}_1 + n_2 \mathbf{a}_2  + n_3 \mathbf{a}_3) 
\end{equation}
mit $n_i$ ganzen Zahlen und $\mathbf{a}_i$ primitiven Einheitsvektoren. Damit ist $\rho(\mathbf{r})$ als Fourier-Reihe darstellbar
\begin{equation}
  \rho(\mathbf{r}) = \sum_{h,k,l} \, \rho_{hkl} \, e^{i \, \mathbf{G}_{hkl} \cdot \mathbf{r}}
\end{equation}
mit $h,k,l$ ganzen Zahlen, den Fourier-Koeffizienten als Integral über die primitive Einheitszelle PEZ
\begin{equation}
\rho_{hkl} = \frac{1}{V_\text{PEZ}} \, \int_\text{PEZ}    \rho(\mathbf{r})\,  e^{-i \, \mathbf{G}_{hkl} \cdot \mathbf{r}} \, dV \label{eq:rezi_rho_hkl}
\end{equation}
und den sogenannten \emph{reziproken Gittervektoren}
\begin{equation}
\mathbf{G}_{hkl} = h \mathbf{b}_1 + k \mathbf{b}_2 + l \mathbf{b}_3 \quad .
\end{equation}
Die $\mathbf{b}_i$ sind die primitiven Einheitsvektoren des reziproken Gitters. $\mathbf{G}_{hkl}$ beschriebt also die Menge aller Gitterpunkte. Jedem dieser Gitterpunkte ist genau ein Fourier-Koeffizient $\rho_{hkl}$ der Streudichte zugeordnet. 

Mit folgender Überlegung können wir die reziproken Einheitsvektoren $\mathbf{b}_i$ bestimmen: die Streudichte ist gitter-periodisch, die Fourier-Koeffizienten $\rho_{hkl}$ aber ortsunabhängig. Damit muss für den Integranden in Gl.~\ref{eq:rezi_rho_hkl} gelten
\begin{equation}
 e^{i \, \mathbf{G}_{hkl} \cdot \mathbf{r}}  =  e^{i \, \mathbf{G}_{hkl} \cdot (\mathbf{r} + \mathbf{R})} \quad ,
\end{equation}
also 
\begin{equation}
 e^{i \, \mathbf{G}_{hkl} \cdot  \mathbf{R}} = 1 \quad \text{und} \quad \mathbf{G}_{hkl} \cdot  \mathbf{R} = 2 \pi \, m
\end{equation}
mit einer ganzen Zahl $m$. Diese Bedingung muss für jede Wahl von $h,k,l$ erfüllt sein. Dies geht nur, wenn
\begin{align}
 \mathbf{b}_i \cdot \mathbf{a}_j & = 2 \pi \, \delta_{ij}  \\
  \mathbf{b}_1 = &\frac{2 \pi}{V_\text{PEZ}} \left(  \mathbf{a}_2 \times \mathbf{a}_3 \right)  \quad \text{und } \, 1,2,3 \, \,\text{ zyklisch} \\
  \text{mit} \quad  V_\text{PEZ} = &  \mathbf{a}_1 \cdot \left(  \mathbf{a}_2   \times \mathbf{a}_3 \right) \quad .
\end{align}
Damit haben wir unser reziprokes Gitter definiert. Die Einheitsvektoren stehen in einem gewissen Sinne senkrecht aufeinander, auch wenn  sie unterschiedliche Einheiten besitzen.


Wir haben nun also zwei Gitter. Im Ortsraum gibt es das Gitter aus dem letzten Kapitel. Die Länge der Vektoren hat die Einheit Meter. Das reziproke Gitter befindet sich im reziproken Raum, auch Impulsraum genannt, weil $\mathbf{p} = \hbar \mathbf{k}$. Die Länge der Vektoren hat die Einheit einer reziproken (oder inversen) Länge, also 1/Meter. Vektoren im reziproken Raum werden auch Wellenvektoren $\mathbf{k}$ genannt. Die Wigner-Seitz-Zelle des reziproken Raums wird (erste) Brillouin-Zone genannt, doch dazu unten mehr.


\section{Beispiele}

\paragraph{Lineare Kette} Eine lineare Kette von Punkten im Abstand $a$ im Ortsraum entspricht im Impulsraum einer Kette von Punkten im Abstand $2 \pi/a$. Das ist die eindimensionale Fourier-Transformation beispielsweise eines zeitlichen Signals. Je größer der Abstand $a$ im Realraum, desto geringer der Abstand der Punkte im reziproken Raum. Daher reziprok!

\paragraph{Kubisch primitives Gitter} In kartesischen Koordinaten sind die primitiven Einheitsvektoren eines kubisch primitiven Gitters im Ortsraum
\begin{equation}
 \mathbf{a}_1 = a \hat{\mathbf{x}} \quad \mathbf{a}_2 = a \hat{\mathbf{y}}
 \quad \mathbf{a}_3 = a \hat{\mathbf{z}} \quad .
\end{equation}
Die Einheitsvektoren des reziproken Gitters sind dann also
\begin{equation}
  \mathbf{b}_1 = \frac{2 \pi}{V_\text{PEZ}} \left(  \mathbf{a}_2 \times \mathbf{a}_3 \right)  = \frac{2 \pi}{a^3} \, a^2  \, \hat{\mathbf{x}} = \frac{2 \pi}{a}  \, \hat{\mathbf{x}}  \quad .
\end{equation}
Das reziproke Gitter eines kubisch primitiven Gitters ist also wieder ein kubisch primitives Gitter, nur eben mit reziproker Gitterkonstante $2 \pi / a$.


\paragraph{Kubisch zentrierte  Gitter} Das kubisch flächenzentrierte Gitter wird im reziproken Raum zu einem kubisch-raumzentrierten Gitter und andersherum. Wir starten vom kubisch raumzentrierten Gitter. In kartesischen Koordinaten sind die primitiven Einheitsvektoren bei einer Kantenlänge $a$ der konventionellen Einheitszelle
\begin{align}
 \mathbf{a}_1 = & \frac{a}{2} \left( - \hat{\mathbf{x}} + \hat{\mathbf{y}} +\hat{\mathbf{z}} \right) \\
  \mathbf{a}_2 = & \frac{a}{2} \left( + \hat{\mathbf{x}} - \hat{\mathbf{y}} +\hat{\mathbf{z}} \right) \\
   \mathbf{a}_3 = & \frac{a}{2} \left( + \hat{\mathbf{x}} + \hat{\mathbf{y}} - \hat{\mathbf{z}} \right)  \quad .
\end{align}
Daraus erhält man die primitiven Einheitsvektoren im reziproken Raum
\begin{align}
\mathbf{b}_1 = & \frac{4\pi}{a} \frac{1}{2} \left( \hat{\mathbf{y}} + \hat{\mathbf{z}}  \right) \\
\mathbf{b}_2 = & \frac{4\pi}{a} \frac{1}{2} \left( \hat{\mathbf{x}} + \hat{\mathbf{z}}  \right) \\
\mathbf{b}_3 = & \frac{4\pi}{a} \frac{1}{2} \left( \hat{\mathbf{x}} + \hat{\mathbf{y}}  \right)  \quad ,
\end{align}
also ein kubisch-flächenzentriertes Gitter, wobei die konventionelle Einheitszelle dann die Kantenlänge $4 \pi / a$ hat.


\section{Brillouin-Zone}

Das reziproke Gitter ist ebenfalls ein Bravais-Gitter. Daher sind alle Definitionen des letzten Kapitels weiterhin gültig. Es gibt also ebenfalls (primitive) Einheitsvektoren und (primitive) Einheitszellen. Das Volumen der primitiven Einheitszelle im reziproken Raum ist gerade reziprok zum Volumen der primitiven Einheitszelle im Realraum, also
\begin{equation}
V_\text{PEZ, reziprok} = \frac{(2 \pi)^3}{V_\text{PEZ, real} }
\quad .
\end{equation}

Ein Unterschied ist, dass im reziproken Raum die Wigner-Seitz-Zelle Brillouin-Zone (BZ) genannt wird, und dass bei der Brillouin-Zone nicht nur das innerste Volumen interessant ist, sondern auch weiter außenliegende Schalen. Alle  Brillouin-Zonen haben aber das gleiche Volumen wie die innerste, erste  Brillouin-Zone.


\begin{marginfigure}

\inputtikz{\currfiledir bz-1d}
\caption{Die ersten drei Brillouin-Zonen in einer Dimension}
\end{marginfigure}

\begin{marginfigure}

\inputtikz{\currfiledir bz-2d}
\caption{Die ersten drei Brillouin-Zonen in zwei Dimensionen. Alle Zonen haben die gleiche Fläche.}
\end{marginfigure}

Punkte hoher Symmetrie in der Brillouin-Zone werden durch große lateinische und griechische Buchstaben bezeichnet ($\Gamma$, L, U, X, etc.). Der $\Gamma$-Punkt ist das Zentrum der ersten Brillouin-Zone, also $\mathbf{k} = 0$. Oft betrachtet man Größen als Funktion des Wellenvektors  $\mathbf{k} $ und damit als Funktion des Ortes in der Brillouin-Zone. Dabei tritt das Problem auf, dass eine skalare Größe als Funktion einer dreidimensionalen Position dargestellt werden müsste. Dies löst man, in dem man als horizontale Achse die Position entlang eines Pfades durch die Brillouin-Zone angibt. Dabei wird der Pfad so gewählt, dass die relevanten Punkte und Symmetrien berücksichtigt werden. Beispiel dazu werden wir im folgenden Kapitel sehen.



\section{Miller'sche Indizes}

In Kristallen spielen oft einzelne Ebenen von Atomen eine Rolle, beispielsweise bei der Röntgen-Beugung. In einer dreidimensionalen Anordnung von Atomen lassen sich aber viele solche Ebenen finden. Die Miller'sche Indizes sind die Methode, um eine Gitterebene eindeutig zu bezeichnen.

Die Miller'sche Indizes werden wie folgt konstruiert:
\begin{enumerate} \setlength{\itemsep}{0pt}
\item Man bestimmt die Achsenabschnitte $x,y,z$ der Ebene im \emph{Realraum}, also beispielsweise $(x,y,z) = (2,3,2)$. Wenn die Achse nicht geschnitten wird, so wird $\infty$ als Achsenabschnitt genommen.
\item Man bildet den Kehrwert jeden Eintrags, hier $(\frac{1}{2},\frac{1}{3}, \frac{1}{2})$.
\item  Man multipliziert alle Einträge mit einer ganzen Zahl, sodass alle Einträge ganzzahlig werden. Normalerweise (s.u.) wählt man die kleinste Zahl, hier also  $6$, sodass wir die Miller'schen Indizes hier erhalten als $(3,2,3)$.
\end{enumerate}
Die Indizes werden in verschiedenen Schreibweisen verwendet:
\begin{description} \setlength{\itemsep}{0pt}
\item $(323)$ runde Klammern ohne  Kommas bezeichnet eine  Schar von parallelen Ebenen, da das Gitter ja translationsinvariant ist.
\item $(\bar{3}23)$ ein Überstrich bezeichnet einen negativen Index
\item $\{hkl\}$ geschweifte\sidenote{spitze Klammern $\braket{hkl}$ analog für Richtungen } Klammern bezeichnen Ebenen gleicher Symmetrie. So entspricht beispielsweise  $\{100\}$ den Ebenen $(100)$, $(010)$ und $(001)$.
\item $[uvw]$ eckige Klammern bezeichnen eine Richtung im Realraum. Aber allein im kubischen Gitter ist $[uvw]$ senkrecht auf der Ebene $(uvw)$.
\end{description}
Die Indizes hängen immer von der Wahl der Einheitszelle ab. In kubischen Gittern wird aber immer die kubisch-primitive Zelle gewählt.

\paragraph{Sonderfall vier Indizes} Im hexagonalen Gitter benutzt man vier Indizes $(hkil)$ mit $i = - (h + k)$, weil die Symmetrie der Ebenen  dann in den Indizes offensichtlicher ist.\sidenote{Beispiel siehe \cite{Hunklinger2014}}

\paragraph{Sonderfall größerer Indizes} Manchmal wird nicht die kleinste Zahl gewählt, um die reziproken Achsenabschnitte ganzzahlig zu machen. Anstatt die Ebene also beispielsweise durch $(100)$ zu beschreiben wird sie als $(200)$ bezeichnet. Damit meint man eine Ebene parallel zu $(100)$, aber in der Mitte der Einheitszelle und nicht an deren Rand. Der Index kodierte also auch die Lage in einer (großen) Einheitszelle. Analog kann auch die Beugungsordnung in einem Streuexperiment kodiert werden. $(200)$  bezeichnet dann die zweite Beugungsordnung eines Reflexes von der $(100)$-Ebene.\sidenote{siehe weiter unten}


\section{Miller'sche Indizes und das reziproke Gitter}

Die Miller'schen Indizes sind die Komponenten eines reziproken Gittervektors, der eine ebene Welle beschreibt, deren Phasenfronten die durch die Miller'schen Indizes beschriebene Ebene sind.

Um dies zu sehen, betrachten wir eine ebene Welle 
\begin{equation}
 A(\mathbf{r}) = e^{i \mathbf{k} \cdot \mathbf{r}}
\end{equation}
mit dem Wellenvektor $\mathbf{k}$ im reziproken Raum, also
\begin{equation}
\mathbf{k} = h \mathbf{b}_1 + k \mathbf{b}_2  +l  \mathbf{b}_3 
\end{equation}
mit $h,k,l$ ganze Zahlen und $ \mathbf{b}_i$ primitive Einheitsvektoren im reziproken Raum. Die Phasenfronten der ebenen Welle sind gegeben durch
\begin{equation}
\mathbf{k} \cdot \mathbf{r} = 2 \pi \, N \quad ,
\end{equation}
wobei die ganze Zahl $N$ die Fronten durchnummeriert. Für den Achsenabschnitt $x_i$ der Phasenfront im Realraum muss gelten (hier entlang der x-Achse, also $i=1$)
\begin{equation}
 \mathbf{k} \cdot (x_1 \mathbf{a}_1) = 2 \pi \, h \, x_1 = 2 \pi \, N \quad .
\end{equation}
Damit ist $h = N / x_1$, also gerade der Miller'sche Index, und analog für $k$ und $l$. Reziproke Gittervektoren, Miller'sche Indizes und ebene Wellen im Realraum sind also das Gleiche. Weiter unten wird relevant werden, dass der Abstand $d$  der Phasenfronten der ebenen Wellen, also die Wellenlänge, reziprok zur Länge $| \mathbf{k} |$ des Gittervektors ist
\begin{equation}
 d = \frac{2 \pi}{| \mathbf{k} |} \quad .
\end{equation}


\section{Bragg-Theorie der Beugung}

Zu Beginn dieses Kapitels hatten wir gesehen, dass die Streuamplitude $\mathcal{A}(\mathbf{K})$, also die Amplitude der gestreuten Welle in Richtung $\mathbf{K} = \mathbf{k} - \mathbf{k}_0$, gerade die Fourier-Transformierte der Streudichte $\rho(\mathbf{r})$ im Realraum ist. Die Bragg-Theorie ist ein anderer Weg, Bedingungen für das Auftreten von konstruktiver Interferenz und damit Peaks im Streubild zu finden. Weiter unten werden wir die verschiedenen Wege zusammenführen.

In welche Richtung $\mathbf{K} = \mathbf{k} - \mathbf{k}_0$ finden sich in Streuexperimenten mit Röntgenstrahlen, Elektronen oder Neutronen starke Peaks, auch Reflexe genannt? Die Bragg-Theorie nimmt an, dass die Atome Ebenen im Abstand $d$ bilden. Die (Materie-)Wellen werden an diesen Ebenen gespiegelt. Wenn der Phasenunterschied passend ist, dann kommt es zur konstruktiven Interferenz und damit einem Peak.


\begin{marginfigure}
\inputtikz{\currfiledir bragg}

\caption{Phasendifferenz bei der Reflexion an zwei Ebenen.}
\end{marginfigure}


Aus geometrischen Überlegungen findet man, dass konstruktive Interferenz genau dann auftritt, wenn die \emph{Bragg-Bedingung} erfüllt ist
\begin{equation}
n \, \lambda = 2 \, d \, \sin \Theta \quad .
\end{equation}
Dabei bezeichnet $n$ die Beugungsordnung, $\lambda$ die Wellenlänge der (Materie-)Welle, $d$ den Abstand der Ebenen und $\Theta$ den Winkel zwischen einfallendem Strahl und Gitter-Ebene. Da es sich um spiegelnde Reflexion handelt, ist dies natürlich auch der Winkel des auslaufenden Strahls mit der Ebene.\sidenote{Aber nicht der zum Lot!} Die Änderung 
$\mathbf{K}$ des Wellenvektors  steht senkrecht auf den reflektierenden Gitterebenen. Für den  Betrag gilt
\begin{equation}
|\mathbf{K}| = |\mathbf{k} - \mathbf{k}_0| = 2 | \mathbf{k}_0| \, \sin \Theta \quad .
\label{eq:rezi_def_k_sin_theta}
\end{equation}
Es gibt in einem Kristall natürlich viele Möglichkeiten, Ebenen zu finden, und damit auch viele Richtungen, die die Bragg-Bedingung erfüllen, also viele Streu-Peaks.
 
 
\section{Laue-Theorie der Beugung} 
 
Die Bragg-Bedingung sagt nur voraus, ob es in eine bestimmte Richtung einen Peak gibt, nicht jedoch dessen Intensität. Die Bragg-Theorie benutzt ja auch nur die Gitter-Periodizität der Streudichte $\rho( \mathbf{r})$, nicht deren genaue Form. Beides geschieht in der  Laue-Theorie.\sidenote{Max von Laue, 1879--1960 }

Die Intensität eines Streu-Peaks ist proportional zum Quadrat der Streuamplitude. Mit Gl.~\ref{eq:rezi_streuamplitude} erhalten wir
\begin{equation}
I(\mathbf{K}) \propto \left| \mathcal{A}(\mathbf{K}) \right|^2 
= \left|   \int_{V_\text{Probe}} \, \rho( \mathbf{r}) \,   e^{- i \,   \mathbf{K} \cdot \mathbf{r}} \, dV \right|^2 \quad .
\end{equation}
Wir wiederholen noch einmal die Schritte vom Anfang des Kapitels und schreiben die Streudichte als Fourier-Summe mit den Koeffizienten $\rho_{hkl}$
\begin{equation}
  \rho(\mathbf{r}) = \sum_{h,k,l} \, \rho_{hkl} \, e^{i \, \mathbf{G}_{hkl} \cdot \mathbf{r}}
\end{equation}
mit $h,k,l$ ganzen Zahlen, den Fourier-Koeffizienten $\rho_{hkl}$ und den  reziproken Gittervektoren
\begin{equation}
\mathbf{G}_{hkl} = h \mathbf{b}_1 + k \mathbf{b}_2 + l \mathbf{b}_3  \quad .
\end{equation}
Im Folgenden lasse ich die Indizes an $\mathbf{G}$ manchmal weg. Damit erhalten wir für die Streu-Intensität 
\begin{equation}
I(\mathbf{K}) \propto \left| \mathcal{A}(\mathbf{K}) \right|^2 
= \left| 
 \sum_{h,k,l} \, \rho_{hkl}
  \int_{V_\text{Probe}}   e^{ i \,   (\mathbf{G}- \mathbf{K} )\cdot \mathbf{r}} \, dV \right|^2 \quad .
\end{equation}
Der Integrand oszilliert schnell mit $\mathbf{r}$ und mittelt sich weg, falls nicht $\mathbf{G} =  \mathbf{K}$. In diesem Fall ergibt das Integral gerade das Probenvolumen $V_\text{Probe}$.

Damit haben wir die  Laue-Streubedingung erhalten
\begin{equation}
\mathbf{G} =  \mathbf{K} \quad .
\end{equation}
Die Änderung des Wellenvektors muss einem Gittervektor entsprechen. Oder andersherum: bei der Beugung am Gitter wird auf den einfallenden Wellenvektor ein Gittervektor addiert.  Die Streuintensität ist in diesem Fall
\begin{equation}
I(\mathbf{K} = \mathbf{G}_{hkl} ) \propto \left| \mathcal{A}(\mathbf{K} = \mathbf{G}_{hkl} ) \right|^2 
= \left| \rho_{hkl} \right|^2   V_\text{Probe}^2  \quad . \label{eq:rezi_laue_peak}
\end{equation}
Ein einziger Fourier-Koeffizient bestimmt also die Intensität des Peaks in Richtung $(hkl)$.


Eine Nebenbemerkung noch zur Form der Peaks im reziproken Raum. Die Lage ist durch $\mathbf{G}$ bestimmt. Die Breite ist endlich, also nicht deltaförmig, da die Probe endlich groß ist. Dies ist analog zur Breite der Beugungspeaks am Strich-Gitter, die ebenfalls mit $1/N_\text{Striche}$ fällt. Im Dreidimensionalen ist die Breite des Peaks also $1/V_\text{Probe}$. Da die Höhe des Peaks proportional zu $ V_\text{Probe}^2$ ist, ist das Integral über einen Peak proportional zu $ V_\text{Probe}$. Dies ist sehr praktisch, da die Intensität des Effekts linear mit der verursachenden Menge Materie gehen sollte, und nicht quadratisch, wie Gl.~\ref{eq:rezi_laue_peak} suggeriert.


\section{Äquivalenz der beiden Bedingungen}

Wir starten von der Laue-Bedingung $\mathbf{G} =  \mathbf{K}$ und leiten daraus die Bragg-Bedingung her:
\begin{equation}
 | \mathbf{K} | =  |\mathbf{k} - \mathbf{k}_0| = 2 | \mathbf{k}_0| \, \sin \Theta = \frac{4 \pi}{ \lambda} \, \sin \Theta = | \mathbf{G} | = n \, | \mathbf{G} |  \quad .
\end{equation}
Die ersten Schritte sind Gl.~\ref{eq:rezi_def_k_sin_theta} und rein geometrische Überlegungen zur Reflexion, also noch nicht die Bragg-Bedingung. Im letzten Schritt haben wir ausgenutzt, dass auch jedes ganzzahlige Vielfache eines Gittervektors wieder ein Gittervektor ist.\sidenote{$\mathbf{G}$ ist ja mit vollem Namen $\mathbf{G}_{hkl}$, also eine Menge von Vektoren}

Wie wir bei den Miller'schen Indizes gesehen hatten, beschreibt jeder Gittervektor eine ebene Welle und damit eine Schar von Ebenen im Abstand 
\begin{equation}
 d = \frac{2 \pi}{|\mathbf{G} | }  \quad .
\end{equation}
 Alles zusammen ergibt dies die Bragg-Bedingung
 \begin{align}
   \frac{4 \pi}{ \lambda} \, \sin \Theta  & = n \frac{2 \pi}{d } \\
   2 d  \, \sin \Theta  & = n \lambda   \quad .
 \end{align}
 

\section{Ewald-Kugel} 
Nur wenige Orientierungen eines Kristalls relativ zum einfallenden Strahl liefern überhaupt Reflexe. Diese Orientierung und die dann sichtbaren Reflexe zu identifizieren ermöglicht die Konstruktion der Ewald-Kugel:
\begin{enumerate} \setlength{\itemsep}{0pt}
\item das Gitter als Punktgitter im reziproken Raum zeichnen

\item den einfallenden Strahl mit dem Wellenvektor $\mathbf{k}_0$ so einzeichnen, dass die Pfeilspitze am Gitterpunkt $(000)$ endet. Dies definiert die Orientierung des Strahls relativ zum Kristall.

\item einen Kreis / eine Kugel um den Anfang von  $\mathbf{k}_0$  mit dem Radius $|\mathbf{k}_0|$ zeichnen. Dies gibt alle Punkte, die $|\mathbf{k}| = |\mathbf{k}_0|$ erfüllen

\item Alle Punkte des reziproken Gitters, die auf dem Kreis / der Kugel liegen, erfüllen die Streu-Bedingung $\mathbf{G} =\mathbf{K} $.
\end{enumerate}

\begin{marginfigure}
\inputtikz{\currfiledir ewald}

\caption{ Konstruktion der Ewald-Kugel}
\end{marginfigure}


In einem endlichen Kristall sind die Gitterpunkte nicht mathematische Punkte, sondern durch die Fourier-Unschärfe zwischen Realraum und reziproken Raum ausgedehnt. Ebenso ist keine (Materie-)Welle exakt deltaförmig im Frequenzraum, da auch hier die Fourier-Unschärfe zwischen Zeit und Frequenz zum Tragen kommt. Für physikalische Systeme gibt es also Punkte, die auf dem Kreis liegen.

Trotzdem sind natürlich viele Ewald-Kugeln denkbar, bei denen allein der Punkt $(000)$ auf der Kugel liegt. Laue-Beugung tritt also nicht immer auf bzw. der Kristall muss dazu genauer orientiert werden. Die Verwendung von breitbandiger Strahlung beispielsweise Bremsstrahlung macht dies einfacher, verliert aber die Möglichkeit, die Gitterkonstante zu messen.

\section{Strukturfaktor}

Bisher haben wir nur das mathematische Gitter und dessen Beugungsbild betrachtet. Jetzt berücksichtigen wir auch die Basis, also insbesondere, wenn diese mehr als ein Atom beinhaltet. Die Kurzfassung ist: Das Gitter bestimmt, in welche Richtung Reflexe auftreten können. Die Basis bestimmt die Intensität dieser Reflexe, die insbesondere auch Null sein kann. Der Grund dafür ist destruktive Interferenz zwischen der Beugung an dem Untergitter aus dem einem Atom mit der an dem anderen.

Gleichzeitig wird hier das 'reziprok' im reziproken Raum auch noch einmal deutlich. Das mathematische Gitter ist im Realraum 'größer' in dem Sinne, dass es durch ganzzahlige Faktoren (also $\ge 1$) vor den primitiven Einheitsvektoren beschrieben wird. Die Basis wird durch Faktoren zwischen Null und Eins beschrieben. Im reziproken Raum geht alles mit dem Kehrwert. Das mathematische Gitter ist dann 'kleiner' als die Fourier-Transformierte der Basis. In Einheiten des primitiven reziproken Vektors des Gitters ist die Basis jetzt verantwortlich für Effekte nicht mehr zwischen Null und Eins, sondern für solche bei ganzen Zahlen $\ge 1$, also der Modulation der Amplitude der Beugungspeaks.

Wir starten von Gl.~\ref{eq:rezi_streuamplitude} und setzen die Definition der Fourier-Komponenten Gl.~\ref{eq:rezi_rho_hkl} ein. Damit erhalten wir
\begin{equation}
\mathcal{A}(\mathbf{K} = \mathbf{G}_{hkl} ) 
=  \rho_{hkl} V_\text{Probe}
=  N_\text{PEZ} \, \int_\text{PEZ}    \rho(\mathbf{r})\,  e^{-i \, \mathbf{G}_{hkl} \cdot \mathbf{r}} \, dV 
\end{equation}
mit der Anzahl der primitiven Einheitszellen $ N_\text{PEZ} =  V_\text{Probe} / V_\text{PEZ}$.
Das Integral über die primitive Einheitszelle teilen wir jetzt auf in eine Summe über die Atome der Einheitszelle und ein Integral über die direkte Umgebung der Atome. Im Endeffekt wird damit wieder über die ganze Einheitszelle integriert. Die alte Ortskoordinate  $\mathbf{r} =  \mathbf{r}' +  \mathbf{r}_\alpha$ schreiben wir als Summe der Position des Atoms  $\mathbf{r}_\alpha$ und der lokalen Koordinate $\mathbf{r}'$ in dessen Umgebung. Damit erhalten wir
\begin{align}
\mathcal{A}(\mathbf{K} = \mathbf{G}_{hkl} ) 
=   &  N_\text{PEZ}  \, \
\sum_\alpha e^{-i \, \mathbf{G}_{hkl} \cdot \mathbf{r}_\alpha} \, \int_{V_\alpha}  
 \rho_\alpha(\mathbf{r'})\,  e^{-i \, \mathbf{G}_{hkl} \cdot \mathbf{r'}} \, dV' \\
 %
 = &
 N_\text{PEZ} \, 
\sum_\alpha f_\alpha ( \mathbf{G}_{hkl} ) \, e^{-i \, \mathbf{G}_{hkl} \cdot \mathbf{r}_\alpha}   \quad .
\end{align}
Das Integral über die Streudichte in der Umgebung des Atoms $\alpha$ ist atom-spezifisch und wird im \emph{Atom-Strukturfaktor} $f_\alpha ( \mathbf{G} )$ zusammengefasst.

Die Koordinaten  $\mathbf{r}_\alpha$  der Atompositionen hängen nur von der Kristallstruktur, also dem Bravais-Gitter ab. Wir schreiben die Position in den primitiven Einheitsvektoren $\mathbf{a}_i$ als
\begin{equation}
\mathbf{r}_\alpha = u_\alpha \, \mathbf{a}_1 + v_\alpha \, \mathbf{a}_2 + w_\alpha \, \mathbf{a}_3
\end{equation}
mit $0 \le u,v,w \le 1$. Zusammen mit der Definition von  $ \mathbf{G}$ erhalten wir dann 
\begin{align}
\mathcal{A}(\mathbf{K} = \mathbf{G}_{hkl} ) 
 = &
  N_\text{PEZ}  \, 
\sum_\alpha f_\alpha ( \mathbf{G}_{hkl} ) \, e^{-2 \pi \, i \, ( h u_\alpha + k v_\alpha + l w_\alpha  ) } \\
 = &
 N_\text{PEZ} \, \mathcal{S}_{hkl}
\end{align}
mit dem \emph{Strukturfaktor} $\mathcal{S}_{hkl} = \rho_{hkl} \, V_\text{PEZ} $.


\section{Beispiel: \ch{CsCl}}

Cäsiumchlorid  (\ch{CsCl}) bildet ein kubisch-primitives Gitter mit einer zweiatomigen Basis, beispielsweise mit dem \ch{Cs}-Atom im Ursprung und dem \ch{Cl}-Atom in der Mitte der Raumdiagonalen. Damit wird der Strukturfaktor 
\begin{equation}
\mathcal{S}_{hkl} = f_\text{\ch{Cs}} \, e^{-2 \pi \, i  \, \mathbf{G} \cdot \mathbf{0}} + f_\text{\ch{Cl}} \, e^{- 2 \pi  \, i \, \frac{1}{2}(h + k + l) }   \quad .
\end{equation}
Die erste Exponentialfunktion ist immer $1$, die zweite ist $+1$, falls die Summe $h + k + l$ gerade ist, und sonst $-1$. Damit ergibt sich
\begin{equation}
  \mathcal{S}_{hkl} = \left\{
  \begin{array}{@{}ll@{}}
    f_\text{\ch{Cs}}  + f_\text{\ch{Cl}}  &\text{falls}\  h+k+l \ \text{gerade} \\
     f_\text{\ch{Cs}}  - f_\text{\ch{Cl}} & \text{falls}\  h+k+l \ \text{ungerade} \\
  \end{array}\right.  \quad .
\end{equation} 
Bei Röntgenstreuung ist $  f_\text{\ch{Cs}}  \approx f_\text{\ch{Cl}}$, also nur jeder zweite Reflex zu sehen. Bei Neutronenstreuung sind die Atom-Strukturfaktoren dagegen deutlich unterschiedlich und alle Peaks zu erkennen.



\section{Beispiel: bcc einatomig und sc zweiatomig}

Ein kubisch-raumzentriertes Gitter  kann als kubisch-primitives Gitter mit einer zweiatomigen Basis gesehen werden. Beides beschreibt die gleiche Lage der Atome im Raum. Allerdings sind es unterschiedliche mathematische Gitter und damit unterschiedliche $\mathbf{G}_{hkl}$. Dies hat scheinbar unterschiedliche Peaks im Beugungsbild zur Folge, was natürlich nicht sein darf.

Die Auflösung findet sich wieder im Strukturfaktor. Die Basis, die benötigt wird, um aus einem  kubisch-primitiven ein kubisch-raumzentriertes Gitter  zu machen, ist wieder die halbe Raumdiagonale, wie im letzten Abschnitt. Im Unterschied zum letzten Abschnitt sind nun aber beide Positionen mit denselben Atomen besetzt. Damit verschwinden alle Peaks bei ungeradem $ h+k+l $. Dies sind gerade die, die den Unterschied zwischen $\mathbf{G}_{bcc} $ und  $\mathbf{G}_{sc} $ ausmachen. Eine ähnliche Bedingung gibt es für das kubisch flächenzentrierte Gitter.


\begin{figure}
\inputtikz{\currfiledir strukturfaktor}
  \caption{Wenn Peaks nach der konventionellen Einheitszelle indiziert werden, dann sind manche in den zentrierten Gittern nicht sichtbar. }
\end{figure}


\section{Auswertung der Pulver-Diffraktometrie}

Aus dem Experiment erhält man die Position der Peaks als Funktion des doppelten Streuwinkels $2\Theta$ bei bekannter Wellenlänge $\lambda$ der Strahlung. Daraus möchte man die möglichen Werte der Länge des Gittervektors $|\mathbf{G}|$ bestimmen und so eine Aussage über das Bravais-Gitter, die Basis und die Gitterkonstante zu treffen. Im Allgemeinen ist dies nicht trivial. In einfachen Fällen, wie dem Beispiel am Anfang des Kapitels, kann man wie folgt vorgehen:

Wir betrachten den reziproken Abstand der Gitterebenen
\begin{equation}
\frac{1}{d} = \frac{|\mathbf{G}|}{2 \pi} = \frac{2 \sin \Theta}{ \lambda}  \quad .
\end{equation}
In kubischen Gittern wird dies
\begin{equation}
 \sqrt{h^2 + k^2 + l^2} = \frac{2 a \, \sin \Theta}{ \lambda}
\end{equation}
mit der Gitterkonstante $a$ der konventionellen Einheitszelle im Realraum, somit also 
\begin{equation}
 h^2 + k^2 + l^2 = \left(\frac{2 a }{ \lambda} \right)^2 \, \sin^2 \Theta  \quad .
\end{equation}
Man versucht also die Position aller Peaks durch eine einzige\sidenote{im Rahmen der Messgenauigkeit} Wahl von $a/\lambda$ und je einen Satz von ganzen Zahlen $(hkl)$ zu beschreiben. Somit erhält man die Gitterkonstante $a$ und aus der An- bzw. Abwesenheit der Peaks den Strukturfaktor und damit das Gitter.


 \newpage
\section{Zusammenfassung}

\textit{Schreiben Sie hier ihre persönliche Zusammenfassung des Kapitels auf. Konzentrieren Sie sich auf die wichtigsten Aspekte und die am Anfang genannten Ziele des Kapitels.}

\vspace*{10cm}



%-------------------

\printbibliography[segment=\therefsegment,heading=subbibliography]
