\renewcommand{\chapterauthors}{Markus Lippitz}
\renewcommand{\lastmod}{21. Oktober 2021}

\chapter{Molekülorbitaltheorie I -- Das aller einfachste Molekül}




\section{Ziele}

\begin{itemize}
\item Sie können die Molekülorbitaltheorie benutzen, um die kovalente Bindung in H$_2^+$ zu erklären und insbesondere die verschiedenen Beiträge  diskutieren.

\item Sie können die Molekülorbitaltheorie von der Valenzbindungstheorie abgrenzen.


\end{itemize}

\section{Überblick}

Die Molekülorbitaltheorie  (MO) baut zunächst neue (Ein-Elektron-) Orbitale, die sich über das ganze Molekül erstrecken. Dann werden in diese Orbitale nach und nach Elektronen eingefüllt, analog zu Mehr-Elektronen-Atomen in der Atomphysik. Dabei wird die Wechselwirkung zwischen den Elektronen zunächst vernachlässigt. Die MO-Theorie macht also andere Näherungen als die Valenzbindungstheorie.

 
 
\section{Das Variationsprinzip}

 
Die Schrödinger-Gleichung
\begin{equation}
 \hat{H} \ket{\Phi} = E_0 \, \ket{\Phi} 
\end{equation}
ist eine Differentialgleichung und nicht immer einfach zu lösen. Hier hilft das Variationsprinzip. Für eine beliebige Wellenfunktion  $\ket{\Psi}$ gilt
\begin{equation}
 E = \frac{\braket{\Psi | H | \Psi}} {\braket{\Psi | \Psi}} \ge E_0 \quad .
 \label{eq:MO_variation}
\end{equation}
Die Mathematik sagt, dass $E$ minimal wird, wenn  $\ket{\Psi}$ die Schrödinger-Gleichung löst. Aber auch wenn $\ket{\Psi}$ keine Lösung der Schrödinger-Gleichung  ist, kann man Gl.~\ref{eq:MO_variation} einfach ausrechnen. Wir probieren  also verschiedene Test-Funktionen durch und versuchen, die Energie nach Gl.~\ref{eq:MO_variation} zu minimieren. Dadurch nähern wir uns der echten Eigenfunktion immer mehr an, die Lösung der Schrödinger-Gleichung ist. Leider wissen wir nicht, ob wir  nicht durch noch bessere Test-Funktionen noch kleinere Werte von $E$ erreichen würde.

Sei die Testfunktion
\begin{equation}
 \ket{\Psi} = c_1 \ket{\phi_1} + c_2 \ket{\phi_2} \label{eq:MO_psi}
\end{equation}
mit normierten  $\ket{\phi_i}$ und reell-wertigen Koeffizienten $c_i$. Damit erhält man
\begin{eqnarray}
\braket{\Psi | \Psi}  &= & c_1^2 + c_2^2  + 2 c_1 c_2 \underbrace{\braket{\phi_1 | \phi_2}}_{= S}\\
\braket{\Psi |  H | \Psi} &=& c_1^2 \underbrace{\braket{\phi_1 |  H | \phi_1 }}_{= H_{11}} +
										c_2^2 \underbrace{\braket{\phi_2 |  H | \phi_2 }}_{= H_{22}} +
								2 c_1 c_2 \underbrace{\braket{\phi_1 |  H | \phi_2 }}_{= H_{12}}  \quad .
\end{eqnarray}
Dabei bezeichnet $S$ das Überlapp-Integral der beiden Wellenfunktionen, und $H_{ij}$ die Matrix-Elemente des Hamilton-Operators. Die Diagonalelemente $H_{11}$ und $H_{22}$ geben die Coulomb-Energie an, die Außerdiagnoalelemente $H_{12} = H_{21}$ die Austausch-Energie\sidenote{mehr zu den Namen im nächsten Abschnitt}. Mit diesen Abkürzungen kann man die Eigen-Energie schreiben als
\begin{equation}
  E = \frac{c_1^2 H_{11} + c_2^2 H_{22} + 2 c_1 c_2 H_{12}}{c_1^2 + c_2^2 + 2 c_1 c_2 S}  \quad . \label{eq:MO_e_variation}
\end{equation}
Für eine minimale Eigenen-Energie $E$ müssen die partiellen Ableitungen nach $c_i$ beide Null sein. Nach ein paar Umformungen findet man zwei Lösungen $E_\pm$ für die minimale Energie $E$ als
\begin{equation}
 \begin{vmatrix}
   H_{11} - E & H_{12} - E \, S \\  H_{12} - E \, S & H_{22} - E \\
 \end{vmatrix}
= 0
\quad
\text{oder} \quad
E_\pm = \frac{H_{11} \pm H_{12}}{1 \pm S} \quad ,
\end{equation}
wobei wir im letzten Schritt angenommen haben, dass $H_{11} = H_{22}$.


\begin{questions} 
\item Vollziehen Sie Gl. \ref{eq:MO_psi} bis \ref{eq:MO_e_variation} nach.
\end{questions}

\section{Das Wasserstoff-Molekül-Ion H$_2^+$}

Als Beispiel für das Variationsprinzip in der Molekülorbitaltheorie betrachten wir das Wasserstoff-Molekül-Ion\sidenote{Dies wäre auch exakt möglich, mittels elliptischer Koordinaten, siehe z.B. Demtröder, Molekülphysik Kap. 2.4.1} H$_2^+$. Es gibt also nur ein Elektron, was das Problem der Elektron-Elektron-Wechselwirkung umgeht.

Wir benutzen wie immer die Born-Oppenheimer-Näherung. Die Kerne bewegen sich also nicht und tragen somit auch nicht zur kinetischen Energie bei. Der Abstand des einzigen Elektrons zu den beiden Kernen sei $r_1$ und $r_2$. Der Hamilton-Operator des Gesamtsystems ist
\begin{equation}
\hat{H} =  - \frac{\hbar^2}{2 m} \nabla^2 - \frac{e^2}{4 \pi \epsilon_0} \frac{1}{r_{1}} - \frac{e^2}{4 \pi \epsilon_0} \frac{1}{r_{2}}
= \hat{H}_1  - \frac{e^2}{4 \pi \epsilon_0} \frac{1}{r_{2}} \quad ,
\end{equation} 
wobei $\hat{H}_1 $ der Hamilton-Operator des Wasserstoff-\emph{Atoms} ist. Die Coulomb-Energie der beiden Kerne untereinander hängt nur vom Kern--Kern--Anstand ab und ist somit eine Konstante, die später zur Gesamtenergie addiert werden wird.

Wir suchen Molekül-Orbitale $\ket{\Psi}$, die mit $\hat{H}$ die Schrödinger-Gleichung lösen, und kennen bereits die Lösungen für $\hat{H}_1$:
\begin{equation}
\hat{H} \ket{\Psi} = E \ket{\Psi} \quad \text{und} \quad 
\hat{H}_1 \ket{\phi} = E_1 \ket{\phi}  \quad .
\end{equation}
Da die beiden Kerne identisch sind, gibt es solche Lösungen $\ket{\phi_2}$ in der gleichen Form aber zentriert um eine andere Kernposition auch für den zweiten Kern. Linearkombinationen von diesen  $\ket{\phi_{1,2}}$ nehmen wir jetzt als Testfunktion $\ket{\Psi}$. Dies nennt man \emph{linear combination of atomic orbitals} (LCAO).

Wir folgen dem oben dargestellten Variationsprinzip und müssen nur die drei Integrale $S$, $ H_{11}$ und $H_{12}$ diskutieren.

\paragraph{Überlappintegral $S$} 
\begin{marginfigure}
\inputtikz{\currfiledir integrals_s}
\caption{Skizze   Überlappintegral $S$. }
\end{marginfigure}
%
Das Integral $S$ beschreibt den räumlichen Überlapp der beiden Atom-Wellenfunktionen, wenn die einen um Kern 1, die andere um Kern 2 zentriert ist:
\begin{equation}
 S = \braket{\phi_1 | \phi_2} = \int \phi_1^\star( \mathbf{r} )  \, \phi_2( \mathbf{r})   \, d\mathbf{r} \quad .
\end{equation}
Dabei bezeichnet $\mathbf{r}$ die Position des Elektrons. Die Wellenfunktion $\phi_i$ ist um den Kern an Position $\mathbf{r}_{i}$ zentriert.\sidenote{Wasserstoff-Wellenfunktionen sind reell-wertig.} Da die $\ket{\phi}$ normiert sind, liegt der Wert von $S$ zwischen $0$ und $1$.





\paragraph{Coulomb-Wechselwirkung $H_{11}$}  
\begin{marginfigure}
\inputtikz{\currfiledir integrals_c}
\caption{Skizze Coulomb-Integral $C$ }
\end{marginfigure}
%
Dieser Term beschreibt die Coulomb-Energie des Elektrons in der atomaren Wellenfunktion $\phi_1$, aber in Gegenwart beider Kerne:
\begin{eqnarray}
H_{11} &= &  \braket{\phi_1 | \hat{H} | \phi_1} = \braket{\phi_1 | \hat{H}_1 | \phi_1}  - \braket{\phi_1 |  \frac{e^2}{4 \pi \epsilon_0} \frac{1}{r_{2}} | \phi_1}  \\
 & = & E_1 - \frac{e^2}{4 \pi \epsilon_0} \int \frac{|\phi_1(\mathbf{r})|^2 }{|\mathbf{r} - \mathbf{r}_2  |} \, d\mathbf{r} = E_1 + C \quad .
\end{eqnarray} 
Das Ergebnis ist die Eigen-Energie des Elektrons im Wasserstoff-\emph{Atom}, korrigiert im ein Überlappintegral der Ladungsdichte ${|\phi_1(\mathbf{r})|^2 }$ um den einen Kern im Coulomb-Potential des anderen Kerns. Der Korrekturterm $C$ ist negativ.




\paragraph{Austausch-Wechselwirkung $H_{12}$} 
\begin{marginfigure}
\inputtikz{\currfiledir integrals_a}
\caption{Skizze Austausch-Integral $A$.}
\end{marginfigure}
%
Die Austausch-Wechselwirkung ist ein rein quantenmechanischer Effekt.
\begin{eqnarray}
H_{12} &= &  \braket{\phi_1 | \hat{H} | \phi_2} = \braket{\phi_1 | \hat{H}_1 | \phi_2}  - \braket{\phi_1 |  \frac{e^2}{4 \pi \epsilon_0} \frac{1}{r_{2}} | \phi_2}  \\
 & = & E_1 \, S - \frac{e^2}{4 \pi \epsilon_0} \int \frac{ \phi_1^\star(\mathbf{r}) \, \phi_2(\mathbf{r})  }{|\mathbf{r} - \mathbf{r}_2  |} \, d\mathbf{r} = E_1 \, S + A \quad .
\end{eqnarray}
Die Austausch-Dichte $\phi_1^\star(\mathbf{r}) \, \phi_2(\mathbf{r})$ ist ähnlich einer Ladungsdichte $|\phi(\mathbf{r})|^2$, nur dass zwei verschiedenen Wellenfunktionen eingehen. Das Elektron wechselt sozusagen zwischen der Zugehörigkeit zu Kern 1 und 2. Der Korrekturterm $A$ ist ebenfalls negativ.



Mit diesen Integralen wird die Gesamtenergie
\begin{equation}
E_\pm = \frac{H_{11} \pm H_{12}}{1 \pm S} = E_1 + \frac{C \pm A}{1 \pm S} \quad .
\end{equation}
Die zugehörigen Molekül-Orbitale sind die symmetrische und die anti-symmetrische Kombination der Atom-Orbitale
\begin{equation}
\ket{\Psi_\pm }= \frac{1}{\sqrt{2 (1 \pm  S)}} \, \left( \ket{\phi_1} \pm \ket{\phi_2} \right) \quad .
\end{equation}

Zur Berechnung der Bindungsenergie nehmen wir jetzt die nur vom Kern--Kern--Abstand $R$ abhängende Coulomb-Energie der Kerne wieder hinzu. 
%
\begin{eqnarray}
 E_\text{Bindung} &=&  E_\text{Molekül} -  E_\text{Atom} \\
  &=&   E_1 + \frac{C \pm A}{1 \pm S} + \frac{e^2}{4 \pi \epsilon_0} \frac{1}{R} - E_1 \\
   &=&\frac{C \pm A}{1 \pm S} + \frac{e^2}{4 \pi \epsilon_0} \frac{1}{R}  = \frac{C' \pm A'}{1 \pm S}  \quad , \label{eq:MO_E_bindung_h2p}
\end{eqnarray}
mit der Definition  von $C'$ und $A'$ wie in Abbildung \ref{fig:MO_H2_integrale_r}.
Numerische Rechnungen zeigen, dass das Überlapp-Integral $S$ keinen entscheidenden Einfluss auf das Ergebnis hat, wir es hier also nicht weiter betrachten müssen.\sidenote{Für H$_2^+$ lassen sich relativ einfache geschlossene Formen für die Integrale angeben, siehe \cite{McQuarrie2008} }

Das Coulomb-Integral $C$ ist für einen  großen Kern--Kern--Abstand $R$ quasi die Energie einer Punkt-Ladung im Potential des anderen Kerns, da die Ausdehnung der Wellenfunktion $\phi_1$ vernachlässigt werden kann. Da $C$ negativ ist, geht $C'$ gegen Null. Für kleine Kern--Kern--Abstände $R$ bleibt $C$ negativ und endlich, da die potentielle Energie eines Elektrons im Wasserstoff-Atom endlich ist. Der zweite Summand von  $C'$ strebt aber mit $1/R$ gegen positiv unendlich. Die Summe der ersten beiden Terme ist also entweder Null oder positiv, so dass kein lokales Minimum zustande kommt.



\begin{marginfigure}
\inputtikz{\currfiledir integrale_von_r}

\caption{Abhängigkeit der Integrale vom Kern--Kern--Abstand $R$. Dargestellt ist 
$C' = C  + \frac{e^2}{4 \pi \epsilon_0} \frac{1 }{R}$ bzw. $A' = A   + \frac{e^2}{4 \pi \epsilon_0} \frac{ S }{R}$. \label{fig:MO_H2_integrale_r}
 }
\end{marginfigure}



Den entschiedenen Beitrag liefert das Austausch-Integral $A$. Für große $R$ ist das Austausch-Integral und auch $A'$ wieder Null. Für kleine Abstände $R$ ist das Austausch-Integral sehr ähnlich dem Coulomb-Integral und endlich negativ. Dazwischen ist es in einem gewissen Bereich von $R$ negativ genug, dass bei positivem Vorzeichen in Gl. \ref{eq:MO_E_bindung_h2p} die Bindungsenergie negativ wird, eine Bindung also zustande kommt.

Damit ist also $\Psi_+$ das bindende Orbital. Da es aus Wasserstoff-1s-Orbitalen zusammengesetzt ist, ist es ein $\sigma$-Orbital. $\Psi_-$ ist ein anti-bindendes $\sigma^\star$-Orbital. Die Skizze zeigt die Gesamt-Energie als Funktion des Kern--Kern--Abstands $R$. Dies wird als \emph{Bindungspotential} bezeichnet. Für das bindende Orbital sind sehr kleine $R$ durch das Pauli-Verbot ausgeschlossen.
Der Bindungsabstand $R_0$ ist der Abstand minimaler Energie. Das Potential kann in seiner Umgebung durch eine harmonisches Parabel-Potential genähert werden. Die Energie $E(R_0)$ bestimmt die Stärke de Bindung, also wieviel Energie aufgebracht werden muss, um die beiden Atome zu trennen. Der zweite Block der Vorlesung zur Spektroskopie von Molekülen beschäftigt sich eigentlich nur mit Methoden, wie die verschiedenen Parameter dieses Bindungspotentials experimentell bestimmt werden können.



\begin{marginfigure}
\inputtikz{\currfiledir potentiale}

\caption{Skizze des Bindungspotentials $E_{\text{Bindung}, \pm}$ vom Kern--Kern--Abstand $R$. Das bindende Potential $E_+$ zeigt ein Minimum bei $R_0$, das anti-bindende Potential $E_-$ hat nur ein Minimum im Unendlichen.}
\end{marginfigure}




\begin{questions} 
\item Die drei Integrale $S$, $C$ und $A$ sind von zentraler Bedeutung. Sie sollten sie sowohl als Gleichung als auch als Skizze darstellen können.

\item Warum sagt man 'Die Austausch-Wechselwirkung ist ein rein quantenmechanischer Effekt' ?
\end{questions}



\section{Das Austausch-Integral für verschiedene Atom-Orbitale}

Wir haben bisher nicht diskutiert, welche Form die Atom-Orbitale $\ket{\phi}$ denn eigentlich haben.
Im Wasserstoff-Molekül-Ion \ch{H2+}  werden es sicherlich s-Orbitale sein (was auch bei der Diskussion der Beiträge angenommen wurde). Bei anderen Orbitalen kann es zum Verschwinden des Austausch-Integrals $A$ kommen, und somit keine Bindung geben.

\begin{marginfigure}
\inputtikz{\currfiledir orbitale_s_p}

\caption{Je nach Art und Orientierung der beteiligten Orbitale kann das Austausch-Integral $A$ auch verschwinden. Die Farben kodieren das Vorzeichen der Wellenfunktion. }
\end{marginfigure}



Ein Beispiel ist das Austausch-Integrals zwischen  einem s-Orbital und einem p$_x$-Orbital, wenn $z$ die Kern--Kern--Achse ist.  Die beiden Keulen des  p$_x$-Orbitals tragen mit unterschiedlichem Vorzeichen zum Austausch-Integral bei und kompensieren sich so. In diesem Fall wäre $A$ Null. Wenn hingehen ein p$_z$-Orbital mit einem s-Orbital überlappt, dann verschwindet das  Austausch-Integral $A$ nicht.


\section{Anschauliche Argumente für eine chemische Bindung}

Kann man anschaulich verstehen, warum das Wasserstoff-Molekül-Ion \ch{H2+}existiert, also energetisch günstiger ist als ein Wasserstoff-Atom und ein freies Proton? Aus meiner Sicht gibt es zwei bis drei Wege.

\paragraph{Elektronen-Dichte-Verteilung} Im symmetrischen Molekülorbital $\Psi_+ \propto \phi_1 + \phi_2$ ergibt sich ein deutlich von Null verschiedener Wert der Elektronendichte $|\Psi_+|^2$ in der Mitte zwischen den beiden Kernen. Diese negative Ladungsdichte schirmt den positiven Kern vom anderen positiven Kern ab. Die Coulomb-Abstoßung der Kerne ist also geringer, als wenn das Elektron in einem s-Orbital um einen Kern alleine  wäre. Im $\Psi_-$-Orbital ist dies nicht mehr der Fall. Hier ist die Elektronen-Dichte zwischen den Kernen geringer, in der Mitte der Strecke sogar exakt Null.
%
\begin{marginfigure}[-50mm]
\inputtikz{\currfiledir wf_bonding}
\caption{ Wellenfunktion (dünne Linie) und Ladungsdichte (dicke Linie) der bindenden Wellenfunktion $\Psi_+$ und der  anti-bindenden Wellenfunktion $\Psi_-$.}
\end{marginfigure}



\paragraph{Teilchen im Kasten}  Man kann das Molekül-Orbital $\Psi_+$ als Kasten für das Elektron sehen, auch wenn die Wände nicht senkrecht und unendlich hoch sind. Die Energie des niedrigsten Zustands in einem eindimensionalen  Kasten-Potential ist proportional zu $1/L$, mit der Kastenlänge $L$. Das Molekül bildet einen größeren Kasten als das Atom, darum sinkt die Energie für das Elektron und es kommt zur Bindung.
%
\begin{marginfigure}
\includegraphics[width=\textwidth]{\currfiledir topf.png}
\caption{Teilchen im Kasten}
\end{marginfigure}

\paragraph{Quantenmechanische Interferenz} Die Ladungsdichte in einem Molekül-Orbital ist $| \phi_1 + \phi_2 |^2$, wenn das Orbital aus den beiden Atom-Orbitalen $\phi_1$ und $\phi_2$ aufgebaut ist. Die Ladungsdichte ist damit \emph{nicht} die Summe der Ladungsdichten der beiden Atom-Orbitale, also nicht $| \phi_1 |^2 +| \phi_2 |^2$. Quantenmechanische Wellenfunktionen interferieren, werden also addiert bevor das Betrags-Quadrat gebildet wird. Dies ermöglicht Auslöschung (im Fall von $\Psi_-$) und konstruktive Interferenz (im Fall von $\Psi_+$), wodurch obiges Elektronendichte-Argument zum Tragen kommt und  die chemische Bindung ermöglicht wird.


\section{Vergleich unserer Rechnung mit der Wirklichkeit}

Unsere Rechnung ergibt ein qualitativ korrektes Bindungspotential für das  Wasserstoff-Molekül-Ion  \ch{H2+}. Quantitativ stimmt sie aber nicht. Experimentell ist der Bindungsabstand $R_0 = 106$~pm, die Rechnung liefert $132$~pm. Experimentell ist die Bindungsenergie $2.5$~eV die Rechnung liefert $1.7$~eV.

Auch erfüllt unser Modell den Virialsatz nicht. Für ein Coulomb-Potential müsste der Mittelwert der potentiellen Energie $\braket{\hat{U}}$ genau $-2$ mal dem Mittelwert der kinetischen  Energie $\braket{\hat{T}}$ sein. In unserem Modell ist der Faktor\footcite{McQuarrie2008} aber nur ca. $-1.6$.

Die Ursache für beides ist, dass unsere Testfunktion $c_1 \ket{\phi_1} + c_2 \ket{\phi_2}$ zwar die Schrödinger-Gleichung löst, aber nicht die geringste Energie in Gl. \ref{eq:MO_variation} liefert. Die Testfunktion ist zu einfach und muss weitere Terme enthalten.

\ \\

\textit{Lesen Sie in \cite{Demtröder_AMP}  den Abschnitt '9.1.3 Improvements to the LCAO ansatz'. Welche Anpassung an der Testfunktion verbessert das Ergebnis? Wie kann man das verstehen? Machen Sie sich hier Notizen dazu.
}

\vspace*{10cm}


\newpage

\section{Zusammenfassung}

\textit{Schreiben Sie hier ihre persönliche Zusammenfassung des Kapitels auf. Konzentrieren Sie sich auf die wichtigsten Aspekte und die am Anfang genannten Ziele des Kapitels.}


\vspace*{10cm}


\printbibliography[segment=\therefsegment,heading=subbibliography]

