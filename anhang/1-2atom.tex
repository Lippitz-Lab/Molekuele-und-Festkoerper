

%\renewcommand{\lastmod}{April 29, 2020}

\chapter{Einatomige Kette dargestellt als zweiatomig}


Wie kann man verstehen, wie eine zweiatomige Kette in eine einatomige übergeht, wenn die beiden Massen gleich werden? Warum bleibt da noch ein optischer Ast übrig?

\paragraph{Einatomige  Kette} Zuerst noch einmal die Dispersionsrelation der einatomigen Kette, Gleichung \ref{eq:phonon_1atom}, mit
Masse $M$, Gitterkonstante $a_1$, Federkonstante $c$


\begin{equation}
\omega = \sqrt{\frac{4 \, c}{M}} \left| \sin \left(\frac{1}{2} \,  k \, a_1  \right) \right|
\end{equation}




\paragraph{Zweiatomige  Kette}  Nun die die Dispersionsrelation der zweiatomigen Kette, Gleichung \ref{eq:phonon_2atom}, mit
Gitterkonstante $a_2$, Federkonstante $c$, Massen $M_1$, $M_2$ und damit der reduzierten Masse $\mu = M_1 M_2 / (M_1 + M_2)$

\begin{equation}
\omega^2 =  \frac{c}{\mu}
\pm c \sqrt{ \frac{1}{\mu^2} - \frac{4}{M_1 M_2}  \sin^2 \left( \frac{1}{2}  k \, a_2 \right) } 
\end{equation}

Jetzt lassen wir die zweiatomige Kette in die einatomige übergehen. Dazu machen wir beide Massen  identisch, also $M_1 = M_2 = M$ und damit $\mu = M/2$. Wir müssen aber auch die Gitterkonstante richtig setzten, nämlich $a_2 = 2 a_1$. Damit wird die Dispersionsrelation
\begin{align}
\omega^2 =  & \frac{2c}{M}
\pm c \sqrt{ \frac{4}{M^2} - \frac{4}{M^2}  \sin^2 \left( \frac{1}{2}  k \, a_2 \right) }  \\
= &\frac{2c}{M}
\pm \frac{2c}{M} \sqrt{ 1 -   \sin^2 \left( \frac{1}{2}  k \, a_2 \right) }  \\
= &\frac{2c}{M}
\pm \frac{2c}{M}   \cos \left( \frac{1}{2}  k \, a_2 \right)   \\
= &\frac{2c}{M} \left( 1 
\pm   \cos \left(   k \, a_1 \right)    \right)
\end{align}

Wir ziehen auf beiden Seiten die Wurzel und wählen ein Vorzeichen des $\pm$.
Zuerst den \emph{Minus-Zweig}, also die akustischen Phononen. Wir benutzen  $\sin(x/2) = \sqrt{ 1-\cos(x) } / \sqrt{2}$ und erhalten
\begin{align}
\omega_{-} =  & \sqrt{\frac{2c}{M}} \, \sqrt{1 - 
  \cos \left(   k \, a_1 \right) }  \\
=  & \sqrt{\frac{4c}{M}} \, 
 \left|  \sin \left( \frac{1}{2}   k \, a_1 \right) \right|
\end{align}
Der akustische Ast stimmt  also völlig mit einatomigen Kette überein.

Nun der\emph{ Plus-Zweig}, also die optischen Phononen. Wir benutzen $\cos(x/2) = \sqrt{ 1+\cos(x) } / \sqrt{2}$ und erhalten
\begin{align}
\omega_{+} =  & \sqrt{\frac{2c}{M}} \, \sqrt{1 + 
  \cos \left(   k \, a_1 \right) }  \\
=  & \sqrt{\frac{4c}{M}} \, 
\left|  \cos \left( \frac{1}{2}   k \, a_1 \right) \right| \\
  =  & \sqrt{\frac{4c}{M}} \, 
\left|  \sin \left( \frac{1}{2}   k \, a_1 + \frac{\pi}{2} \right)  \right| \\
    =  & \sqrt{\frac{4c}{M}} \, 
\left|  \sin \left( \frac{1}{2}   (k + G)  \, a_1 \right)  \right|
\end{align}
mit $G   =\pi / a_1 = 2 \pi / a_2$, also dem kürzesten reziproken Gittervektor der \emph{zweiatomigen} Kette. Auch optischer Ast stimmt mit einatomiger Kette überein, ist aber um $G$ verschoben. 




  \begin{figure}
\inputtikz{\currfiledir 1atom-2atom-sketch}
\caption{Dispersionsrelationen ein- und zweiatomiger Ketten identischer Massen.}
  \end{figure}

Die Dispersionsrelation der zweiatomigen Kette mit identischen Massen entsteht aus der einatomigen,  indem man die Grenze der zweiatomigen Brillouin-Zone berücksichtigt. Werte außerhalb der 1. BZ enthalten keine neuen Schwingungsmuster. Wenn man bei $k=\pi/a_2$ die BZ verlässt, dann kommt man bei $k = - \pi/a_2$ wieder rein. Dadurch bildet der überstehende Teil der einatomigen Dispersionsrelation den optischen Ast der zweiatomigen Dispersionsrelation. Dies ist die Konsequenz des $+G$: in einem Gitter gilt die Impulserhaltung nur bis auf einen additiven reziproken Gittervektor $\mathbf{G}$.

Weil die allermeisten Dispersionsrelation nur von $|k|$ abhängen, also symmetrisch um den $\Gamma$-Punkt sind, kann man das 'hier raus, da rein' auch als 
Zurückfalten verstehen. Die einatomige Dispersionsrelation wird an der Grenze der zweiatomigen BZ zurückgefaltet.\sidenote{Weil $|\sin (-x)| = |\sin (x)|$ und auch $-G$ ein reziproker Gittervektor ist.}
 Dies spielt eine wichtige Rolle bei der Dispersionsrelation von Elektronen im Gitter (engl: folded zone scheme). Dort führt die 'empty lattice approximation'  zur Bandlücke der Elektronen im Festkörper.
 
Was passiert hier? Wenn wir ein Gitter mit Gitterkonstante $a_2 = 2 a_1$ annehmen, dann reduzieren wir damit die Translationsinvarianz des Raumes. Wir können nicht mehr um $a_1$ verschieben, ohne dass sich etwas ändert. Dies hat direkte Konsequenz für die Impulserhaltung. Der kleinste reziproke Gittervektor ändert sich von $2 \pi / a_1$ auf $ \pi / a_1$. Damit werden mehr Wellenvektoren als äquivalent angesehen, und dadurch kommt der optische Ast ins Spiel.







%-------------------

