

%\renewcommand{\lastmod}{April 29, 2020}

\chapter{Vergleich Einstein--Debye}


Warum passt das Einstein-Modell der Wärmekapazität nicht bei tiefen Temperaturen, das Debye-Modell aber schon?




Die Wärmekapazität  (Gl. \ref{eq:wk_u_allg} zzgl. einem Faktor 3 für die drei Dimensionen) ist
\begin{equation}
 C_V = \left( \frac{\partial U}{\partial T} \right)_{V = \text{const.}}
 =  3 \, \frac{\partial }{\partial T}  \int_{\omega=0}^{\infty} \, \hbar \omega \, \braket{ n (\omega, T) } \, D(\omega) \, d\omega  \quad .
\end{equation}
%
\begin{marginfigure}
\inputtikz{\currfiledir BE_stat_tvar}
\caption{Die Bose-Einstein-Verteilung für Phononen-Energien in Einheiten von $k_B T_0$ für verschiedene Temperaturen $T_j = j \, T_0$.}
\end{marginfigure}
%
Im Integral hängt nur die Besetzungsfunktion $ \braket{ n (\omega, T) } $ von der Temperatur ab. Deren Ableitung nach $T$ ergibt
\begin{equation}
\frac{\partial  }{\partial T} \braket{n (\omega, T)} = 
\frac{\hbar \omega}{k_B T^2} \, \frac{e^{\hbar \omega / k_B T}}{ \left( e^{\hbar \omega / k_B T} -1 \right)^2} \label{eq:ED_n_abl} \quad .
\end{equation}
Dieser Term beschreibt also, welche Zustände zur Wärmekapazität beitragen. Wärme aufnehmen bedeutet ja, die Besetzung $n$ eines Zustandes zu erhöhen. Auch wenn die Besetzungsfunktion $ \braket{ n (\omega, T) } $ 
bei allen Frequenzen $\omega$ von der Temperatur abhängt, so sieht man in Abbildung \ref{fig:ED-Tabl} schon, dass die größte Änderung von $n$ mit $T$ bei Phononen-Energien $\hbar \omega \lesssim k_B T$ stattfindet. Zur Wärmekapazität tragen daher hauptsächlich Zustände bei, deren Energie kleiner ungefähr der thermischen Energie ist. Zustände mit größerer Energie tragen quasi nicht bei.
%
\begin{marginfigure}
\inputtikz{\currfiledir BE_stat_Tableitung}

\caption{Die Änderung der  Bose-Einstein-Verteilung mit der Temperatur $T$  für verschiedene Temperaturen $T_j = j \, T_0$. \label{fig:ED-Tabl}}
\end{marginfigure}

Im Debye-Modell ist die Zustandsdichte (Gl. \ref{eq:wk_debye_dos})
\begin{equation}
D(\omega) d\omega = \left( \frac{L}{2 \pi} \right)^3 \,     \frac{ 4 \pi \, \omega^2 }{v^3}   \, d\omega  \quad .
\end{equation} 
Damit gibt es ein Kontinuum an Zuständen zwar mit niedriger aber trotzdem von Null verschiedenere Dichte $D$ bis hinab zu tiefen Frequenzen, tiefen Energien pro Phonon. Daher stehen bei allen Temperaturen Zustände zur Verfügung, die zur Wärmekapazität beitragen. Die Wärmekapazität fällt mit der Temperatur, weil $D$ mit $\omega$ fällt, aber nur wie ein Polynom, nicht exponentiell.

Im Einstein-Modell ist die Zustandsdichte
\begin{equation}
D(\omega) \propto \delta(\omega_0 - \omega) \quad .
\end{equation}
Zur Wärmekapazität trägt nur \emph{ein} Zustand bei. Wenn dessen Energie $\hbar \omega_0 \gg k_B T$ ist, dann ändert sich bei dieser Energie die Besetzung $n$ quasi nicht mit der Temperatur. Die Wärmekapazität fällt darum exponentiell, so wie die $\partial n / \partial T$ bei festem $\omega$ mit $T$ fällt.
Wenn man  obige Gleichung \ref{eq:ED_n_abl} bei $\omega = \omega_0$ auswertet, und dann noch mit der Energie pro Phonon $\hbar \omega_0$ multipliziert, dann hat man schon die Wärmekapazität im Einsteinmodell (Gl. \ref{eq:wk_einstein_full}) hergeleitet.



%
%-------------------

