%\renewcommand{\lastmod}{April 29, 2020}

\chapter{Bindungen in Festkörpern}





\section{Ziele}

\begin{itemize}
\item Sie dewe

\item S.

\end{itemize}


%\begin{figure}
%\inputtikz{\currfiledir fig_bodipy}
%  \caption{Absorptions- und Fluoreszenz-Spektrum des Farbstoffs BODIPY  (\href{https://www.thermofisher.com/de/de/home/life-science/cell-analysis/labeling-chemistry/fluorescence-spectraviewer.html?SID=srch-svtool&UID=10001moh}{thermofischer.com}).}
%\end{figure}



\section{Überblick}



Wie bei den Molekülen  sind es auch in den Festkörpern die Elektronen der Atome, die die Bindung bewirken. Man kann die verschiedenen Bindungen also nach der Verteilung der Elektronen unterscheiden. Im Gegensatz zu Molekülen bilden sich Festkörper auch aus Atomen mit abgeschlossenen Schalen. Die schwache van-der-Waals-Kraft bildet beispielsweise Edelgaskristalle bei niedrigen Temperaturen. In Ionenkristallen wird ein Elektron transferiert und die Coulomb-Anziehung  bestimmt die Bindung. Die kovalente Bildung in Festkörpern ist analog zur der in Molekülen. Als Extremfall gibt es die metallische Bindung, bei der manche Elektronen über den ganzen Kristall gleichmäßig verteilt sind. Ein Sonderfall ist die Wasserstoff-Brückenbindung, die aber in der Biologie eine große Rolle spielt. Im Folgenden werden wir diese Bindungen besprechen.

XXX Fig Hellwege


\section{Van-der-Waals-Bindung}

Die Van-der-Waals-Bindung ist sehr schwach (meV pro Atom), aber immer vorhanden. Sie wird also nur dann sichtbare, wenn andere Bindungskräfte nicht zum Tragen kommen, beispielsweise in Kristallen von Edelgas-Atomen.

Die Verteilung der Elektronen um einen Atomkern ist kein starres Gebilde. Die Elektronendichte-Verteilung fluktuiert und in einem Augenblick kann es ein netto Dipolmoment $\mathbf{p}_1$ bei Atom $1$ geben. Dieses Dipolmoment ist verknüpft mit einem elektrischen Feld, dessen Amplitude $E_1 \propto p_1 / r^3$ ist.\sidenote{Die Richtungsabhängigkeit ignorieren wir hier.} Ein Atom $2$ in der Entfernung $r$ wird dann durch dieses Feld polarisiert und es bildet sich ein induziertes Dipolmoment $p_2 = \alpha E_1 \propto 1/r^3$. Das Wechselwirkungspotentail $\phi(r)$ ist damit das Potential von $\mathbf{p}_2$ im Feld $\mathbf{E}_1$, 
\begin{equation}
 \phi(r) = - \mathbf{p}_2 \, \mathbf{E}_1 \propto - \frac{B}{r^6}
\end{equation}
wobei die Konstante $B$ positiv und für das jeweilige Element charakteristisch ist.

Bei kleine Atom--Atom--Abständen kommt die abstoßende Kraft aufgrund des Pauli-Verbots hinzu. Oft modelliert man sie als $A/r^{12}$. Insgesamt ergibt sich damit das \emph{Lenard-Jones-Potential}
\begin{equation}
\phi(r) = \frac{A}{r^{12}} - \frac{B}{r^{6}} = 4 \epsilon \left[ \frac{\sigma}{r^{12}} -  \frac{\sigma}{r^{6}} \right]
\end{equation}
wobei $\epsilon$ die Tiefe und $\sigma$ den Nulldurchgang des Potentials, sozusagen den Radius des Atomrumpfes, bestimmt. Der Gleichgewichtsabstand ist $r_0 = 2^{1/6} \sigma \approx 1.12 \sigma$.

In einem Kristall wirkt dieses Potential zwischen allen Paaren von Atomen, also ist die Bindungsenergie 
\begin{equation}
U_B = \frac{1}{2} \sum_{m,n}^N \phi_{n,m} = 
2 N \epsilon \sum_{m=1,   \neq n}^N 
\left[ 
\frac{\sigma}{r_{nm}^{12}} -  \frac{\sigma}{r_{nm}^{6}} 
\right]
\end{equation}
Der Faktor $1/2$ kommt daher, dass man in der ersten Summe alle Bindungs-Paare doppelt zählt. Die zweite Summe läuft nur noch über einen Summanden, da der Kristall translationsinvariant ist und es daher ausreicht, von einem Atom ausgehend alle Bindungen zu betrachten.

Dies lässt sich weiter vereinfachen, in dem man den Atom--Atom--Abstand $r_{nm}$ schreibt als $r_{nm} = \rho_{nm} \, a$  mit der Gitterkonstanten $a$ im kubischen flächenzentrierten (fcc) Gitter:
\begin{align}
U_B = & 
2 N \epsilon 
\left[ 
\frac{\sigma^{12}}{a^{12}}
\sum_{m=1,   \neq n}^N  \frac{1}{\rho_{nm}^{12}} 
-
\frac{\sigma^{6}}{a^{6}}
\sum_{m=1,   \neq n}^N  \frac{1}{\rho_{nm}^{6}} 
\right] \\
\approx & 
2 N \epsilon 
\left[ 
12.13 \; \frac{\sigma^{12}}{a^{12}}
-
14.45 \; \frac{\sigma^{6}}{a^{6}}
\right] 
\end{align}
Das Minimum der Bindungsenergie im Kristall findet sich nun bei $r_0 \approx 1.09 \sigma$, also etwas näher als im vd-Waals-'Molekül'. Dieser Wert wird auch experimentell für die schwereren Edelgasatome gefunden.

\section{Ionische Bindung}

Bei der Ionischen Bindung dominiert die Coulomb-Anziehung zwischen unterschiedlich geladenen Ionen. Die van-der-Waals-Wechselwirkung vernachlässigen wir also. Am Beispiel von \ch{NaCl} diskutieren wir die einzelnen Beiträge zur Bindungsenergie.

Zunächst müssen wir beide Atome ionisieren
\begin{align}
 \ch{Na} + 5.14\text{ eV} \rightarrow  & \ch{Na+} + \ch{e-} \\
  \ch{e-}  + \ch{Cl} \rightarrow  & \ch{Cl-} + 3.16\text{ eV}
\end{align}
Wir müssen also netto 1.53 eV aufwenden, um ein Ionenpaar herzustellen.

Dann bringen wir die beiden Ionen aus dem Unendlichen zusammen, bis zum experimentell gefundenen Bindungsabstand von $R_0 = 2.81$~\AA. Die Coulomb-Energie bei diesem Abstand beträgt -5.1~eV. Insgesamt gewinnen wir also durch Bildung eines \ch{NaCl}-'Moleküls' 3.57~eV. Dabei ist aber der abstoßende Teil des Biundungspotentials und die Wechselwikrung mit allen andeern Ionen des kristall noch nicht berücksichtugt.

Analog zur van-der-Waals-Wechselwirkung setzen wir also wieder als Potential zwischen einem Paar von Ionen an
\begin{equation}
 \phi(r_{nm}) = \pm \frac{e^2}{4 \pi \epsilon_0 \, r_{nm}} + \frac{B}{ r_{nm}^{12}}
\end{equation}
Das wechselnde Vorzeichen berücksichtigt die abwechselnd anziehende und abstoßende Wechselwirkung, je nach Ladung des Ions. Wir bilden wieder die Summe über alle Paare
\begin{equation}
U_B = N \sum_{m \neq n} \, \phi(r_{nm})  = N \sum_{m \neq n} \, \phi(\rho_{nm} \, a) 
\end{equation}
wobei $N$ nun die Anzahl der Ionen einer Sorte bezeichnet und so der Faktor 2 überflüssig wird. Wir haben wieder die Abstände durch die Gitterkonstante ausgedrückt. Nun machen wir die Annahme, dass die Abstoßende Wechselwirkung kurzreichweitig ist und nur $z$ Ionen im Abstand $a$ beitragen. Damit wird  
\begin{align}
U_B = & + z \frac{ B}{ a^{12}} - N \sum_{m \neq n} \, \pm \frac{e^2}{4 \pi \epsilon_0 \, \rho_{nm} \, a}  \\
= & + z \frac{ B}{ a^{12}} - N \frac{e^2}{4 \pi \epsilon_0 \,  \, a}  \, \alpha
\end{align}
mit der \emph{Madelung-Konstante} $\alpha$
\begin{equation}
 \alpha = \sum_{m \neq n} \, \frac{\pm}{\rho_{nm} }  
\end{equation}
Die Madelung-Konstante hängt nur von der Art des Gitters ab. Ihre Berechnung erfordert allerdings ein paar Tricks, da die Reihe insbesondere im Dreidimensionalen nicht gut konvergiert. 

\begin{marginfigure}

\begin{tabular}{ll}
kubisch flächen-z. & $\alpha \approx 1.747$ \\
kubisch raum-z. & $\alpha \approx 1.763$ \\
Diamant & $\alpha \approx 1.64$ \\
\end{tabular}
\caption{Die Madelung-Konstante $\alpha$ hängt nur schwach vom Gitter-Typ ab.}
\end{marginfigure}

Am Gleichgewichts-Abstand ist die Ableitung nach der Gitterkonstante Null, wodurch sich $B$ bestimmen lässt. Damit ergibt sich für die Bindungsenergie pro Ion einer Sorte
\begin{equation}
\frac{U_B}{N} = \frac{e^2 \, \alpha }{4 \pi \epsilon_0 \, R_0} \, \left( 1- \frac{1}{12} \right) = U_\text{Coulomb} \, \alpha \, \left( 1- \frac{1}{12} \right) 
\end{equation}
Für \ch{NaCl} ergibt sich so ein Wert von -8.25~eV, nah am experimentell gefundenen Wert von -8.15~eV.


\section{Kovalente Bindung}

Siehe Teil I zur Theorie der Molekülbindung.


\section{Metallische Bindung}

Die metallische Bindung kann als extreme Form der kovalenten Bindung gesehen werden. Die beteiligten Elektronen sind über den gesamten Kristall delokalisiert und bilden ein freies Elektronengas. Die Bindungsenergie liegt im Bereich von 1--5~eV pro Atom.

\section{Wasserstoffbrückenbindung}

In einer Wasserstoffbrückenbindung bildet ein Wasserstoff-Atom nicht eine kovalente Bindung zu einem anderen Atom, sondern zu zwei Atomen. Dies ist natürlich keine gewöhnliche kovalente Bindung. Dazu fehlt dem Wasserstoff  Bindungselektron.

Die Wasserstoffbrückenbindung entsteht, wenn das H-Atom stark kovalent an einen Bindungspartner gebunden ist. Bei geht das Elektron des H-Atoms fast vollständig auf den Partner über und es verbleibt quasi ein Proton. Dies wirkt anziehend auf andere negativ geladene Bindungspartner. Aus räumlichen Gründen\sidenote{Das Proton ist viel kleiner als alle Atome mit Elektronen} kann jeweils nur ein weiterer Bindungspartner wechselwirken. Die Bindung ist daher auch oft asymmetrisch, also \ch{A}$-$\ch{H}$--$\ch{B}. Typische Bindungsenergien sind 0.2~eV, bei \ch{HF} kann aber auch 1.6~eV erreicht werden.

Diese Bindung spielt in der Biologie eine große Rolle. Proteine sind typischerweise durch viele Wasserstoffbrückenbindung verbunden. Jede einzelne Bindung ist schwach, nicht viel stärker als $k_B T$, und kann so einfach geöffnet werden, um eine Funktionalität zu erzeugen. Gleichzeitig sind alle Bindungen zusammen stark, ähnlich einem Klettverschluss.


%-------------------




\printbibliography[segment=\therefsegment,heading=subbibliography]
