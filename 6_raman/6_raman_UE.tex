\renewcommand{\lastmod}{\today}

\chapter{Raman-Spektroskopie}


\section{Ramanspektroskopie}
%ML14


\begin{itemize}

	\item[\textbf{(a)}] Erklären Sie wie Raman-Streuung funktioniert. Benutzen Sie sowohl eine quantenmechanische Beschreibung (mit Übergängen zwischen Zuständen) als auch eine klassische (mit oszillierenden Ladungen).

	\item[\textbf{(b)}] Wie lauten die Auswahlregeln für Übergänge im Schwingungs- und im Rotations-Raman-Spektrum eines zweiatomigen Moleküls? Erläutern Sie anschaulich, wie diese zustande kommen. Skizzieren Sie das gesamte Raman-Spektrum eines zweiatomigen Moleküls und benennen Sie die Linien bzw. Zweige.

	\item[\textbf{(c)}] Welche der folgenden Moleküle zeigen Raman-aktive Schwingungsmoden (jeweils mit Begründung): N$_2$, C$_2$H$_4$, CH$_3$OH, HD, CCl$_4$, CS$_2$, SO$_2$, NH$_3$, BeCl$_2$ (linear), CH$_3$COCH$_3$?
	
\end{itemize}


\section{Kernspineinflüsse auf Rotations-Ramanspektren}
% AK17

a) Der gesamte Kernspin Iges eines zweiatomigen homonuklearen Moleküls mit IA = IB = I ergibt sich durch Vektoraddition zu
Iges =2I,2I-1,...,0. (1) Für ganzzahlige I sind die Kernspinzustände mit
Iges =2I,2I-2,...,0 (2) symmetrisch gegen Kernaustausch, die anderen antisymmetrisch. Bei halbzahligen I dagegen
sind die symmetrischen Zustände
Iges =2I,2I-2,...,1, (3)
alle übrigen sind wiederum antisymmetrisch. Zeigen Sie, dass man aus der (2Iges + 1)-fachen Entartung jedes Kernspinzustandes für das Verhältnis der statistischen Gewichte von antisymmetrischen (ga) zu symmetrischen Zuständen (gs) die allgemeine Formel
ga= I (4) gs I+1
erhält.

b) Wasserstoff 1H besitzt einen Kernspin von I = 21 und Sauerstoff 16O einen Kernspin von I =
0. Die elektronische Wellenfunktion des Grundzustandes von 1H2 hat positive, die von 16O2
negative Parität (Ortsanteil). Die Schwingungswellenfunktion besitzt in beiden Fällen positive
Parität. Für die Parität der Rotationswellenfunktionen gilt (-1)J mit der Rotationsquantenzahl
  J. Wie verhalten sich jeweils die Intensitäten aufeinanderfolgender Rotationslinien (d.h. solcher 1 16
Linien mit geradem zu solchen mit ungeradem J) für H2 bzw. O2?


\section{Raman- und IR-Aktivität II}
% AK 17
s
 Im Infrarot-Absorptionsspektrum und im Raman-Spektrum eines Moleküls A2B2 findet man die fol- genden Linienintensitäten:
 cm-1 IR
3374 -
3287 sehr stark; PR-Struktur 1973 -
729 sehr stark, PQR-Struktur 612 -
Raman stark
-
sehr stark -
schwach
  Auf welche Molekülstruktur schließen Sie aus diesen Daten? Weisen Sie die beobachteten Spek- trallinien den einzelnen Schwingungsmoden zu und berücksichtigen Sie die bekannten Schwinungs- frequenzen häufig vorkommender Bindungen, um das Molekül A2B2 schließlich zu identifzieren.


\section{Raman- und IR-Spektren von CO2}
% AK17

In untenstehender Abbildung sehen Sie eine schematische Darstellung des Schwingungs-IR- und des Schwingungs-Raman-Spektrums von CO2, wobei die Rotationslinien vernachlässigt wurden. Weisen Sie den jeweiligen Linien die zugehörigen Normalschwingungen des CO2-Moleküls zu und begründen Sie warum diese jeweils IR- oder Raman-aktiv sind.


\section{Raman-Spektroskopie}
% JK19
Betrachten Sie folgendes Ramanspektrum:
a) Bei welcher Temperatur wurde dieses Spektrum aufgenommen? b) Ordnen Sie den Peaks die jeweiligen Schwingungen zu.
 
c) Bei der Anregung eines Moleküls mit einem Laser einer bestimmten Wellenlänge erhalten Sie folgendes Fluoreszenzspektrum:
 Wie können Sie überprüfen, ob der Peak in der Fluoreszenz ein Raman-Peak oder eine Charakteristik der Fluoreszenz ist?