
  
\begin{tikzpicture}
%\useasboundingbox (-1.3,-1.2) rectangle (11.2,4.7);
%	\draw (-1.5,-2.5) rectangle (3.5,7.5);
  \tikzmath{\x1 = 0;  \dx1 = 3;  \dx2 = 2.5; \wv = 2;  \brot = 0.05;  \bs = 0.8;  \ylu = -1.7; \ylo = -1; \arrowoffs = 0.12; \ytop = 5;}


 % Zustände
  \foreach \vib in {0,...,1}
  {
  		\tikzmath{\b = \wv * (\vib + 0.5);  }
    	\draw[line width=1pt] (\x1,\b) -- ++ (\dx1,0) node[anchor=west]{\vib};
    	
    	  \foreach \rot in {1,...,4}
     {
    	 		\tikzmath{\br1 = \b + \brot * \rot * (\rot +1);  }
  	\draw[line width=0.5pt] (\x1,\br1) -- ++ (\dx2,0) node[anchor=west]{\footnotesize \rot};
 
    	}
       }
  
      	\draw[line width=1pt, dashed] (\x1,\ytop) -- ++ (\dx1,0) node[anchor=west]{};
      	
   \node [anchor=west] at (\dx2,4.3) {$J$};
   \node [anchor=west] at (\dx1,4.3) {$\nu$};

  % Übergangs-Pfeile
   \foreach \a in {2,...,3}
   {
  		\tikzmath{\xa1 = \x1 + (2 * \a  -2) * \arrowoffs;} 		 
  		\tikzmath{\xa2 = \x1 + (2 * \a +0.5 -2) * \arrowoffs;} 		 
  		 \tikzmath{\br1 = \wv * (0 + 0.5) + \brot * \a * (\a +1);  
  				 \br2 = \wv * (1 + 0.5) + \brot * (\a-1) * (\a - 2);  }
    	\draw[->] (\xa1,\br1) -- node (P\a) {}  (\xa1,\ytop) ;
    	\draw[->] (\xa2,\ytop) -- node (P\a) {}  (\xa2,\br2) ;
  }
 
    \foreach \a in {0,...,1}
   {
   		\tikzmath{\xa1 = 0.8 + \x1 + (2 * \a  +2 ) * \arrowoffs;} 		 
  		\tikzmath{\xa2 = 0.8 + \x1 + (2 * \a +0.5 +2 ) * \arrowoffs;} 		 
  		 \tikzmath{\br1 = \wv * (0 + 0.5) + \brot * \a * (\a +1);  
  				 \br2 = \wv * (1 + 0.5) + \brot * (\a) * (\a +1);  }
    	\draw[->] (\xa1,\br1) -- node (P\a) {}  (\xa1,\ytop) ;
    	\draw[->] (\xa2,\ytop) -- node (P\a) {}  (\xa2,\br2) ;
  }

    \foreach \a in {0,...,1}
   {
   		\tikzmath{\xa1 = 1.6+ \x1 + (2 * \a  +2 ) * \arrowoffs;} 		 
  		\tikzmath{\xa2 = 1.6 + \x1 + (2 * \a +0.5 +2 ) * \arrowoffs;} 		 
  		 \tikzmath{\br1 = \wv * (0 + 0.5) + \brot * \a * (\a +1);  
  				 \br2 = \wv * (1 + 0.5) + \brot * (\a +2) * (\a +3);  }
    	\draw[->] (\xa1,\br1) -- node (P\a) {}  (\xa1,\ytop) ;
    	\draw[->] (\xa2,\ytop) -- node (P\a) {}  (\xa2,\br2) ;
  }
  
    \node [anchor=center, name=nl] at (\x1 +2.5 * \arrowoffs,0.6) {$-2$};
    \node [anchor=center, name = nm] at (0.8 + \x1 +2.5 * \arrowoffs,0.6) {$0$};
    \node [anchor=center, name = nr] at (1.6 + \x1 +2.5 * \arrowoffs,0.6) {$+2$};
    \node [anchor=center] at (2.7,0.6) {$\Delta J$};
%
  
   % Energie-Achse
	\draw[->] (0,\ylu) -- (3,\ylu) node[anchor = north east,xshift=2mm]{  $\Delta$E Stokes};  
 	\draw[-] (0,-1.6) -- (0,-1.8) node[below]{0};  

	\draw[->] (-0.4,0) -- (-0.4,5.5) node[anchor = south east, rotate=90 ]{ Energie des Zustands};  
	\draw[-] (-0.3,0) -- (-0.5,0) node[anchor = east]{0};  

  
  % linien
   \foreach \a in {0,...,3}
   {
   		\tikzmath{\b =  \brot * ( 4 * \a + 6); \x0 = \wv;}

    	\draw[line width = 0.1pt] (\x0+\b,\ylu) -- (\x0+\b,\ylo) node (linieR\a) {};
     	\draw[line width = 0.1pt] (\x0-\b,\ylu) -- (\x0-\b,\ylo) node (linieP\a) {};
}

     	\draw[line width = 1pt] (\wv,\ylu) -- (\wv,\ylo) node (linieQ) {};


     	
  \draw[->, blue] (nl) .. controls (linieP1.north)  ..  (linieP1);
  \draw[->, blue] (nm) .. controls  (linieQ.north)  ..  (linieQ);
  \draw[->, blue] (nr) .. controls (linieR1.north)  ..  (linieR1);
 
% 
\draw[white, line width= 2pt]  (-0.5,4.6 ) -- ++(3, 0 );
\draw[line width=0.3pt] (-0.4,4.6 +0.02) -- ++ (30:0.07) -- ++ (-150:0.14);
\draw[line width=0.3pt] (-0.4,4.6 -0.02) -- ++ (30:0.07) -- ++ (-150:0.14);
\end{tikzpicture}
