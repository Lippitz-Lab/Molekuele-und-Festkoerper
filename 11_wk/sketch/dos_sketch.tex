\documentclass[margin=0mm]{standalone}
\usepackage{tikz}
\usepackage{pgfplots}
 \pgfplotsset{compat=newest}


\usepackage{currfile,hyperxmp}
\usetikzlibrary{math,calc,matrix,fit,positioning,intersections}
\usetikzlibrary{quotes,angles}


\begin{document}



  
\begin{tikzpicture}[font=\footnotesize]

%\useasboundingbox (-1.3,-1.2) rectangle (11.2,4.7);
%	\draw (0,0) rectangle (5,4);

  
%\clip (-26mm,-16mm) rectangle  (26mm, 16mm);  
   
 \foreach \u in {-16,-15,...,16}{%  
     \foreach \v in {-10,-9,...,10}{%  
          \draw[fill=gray]  (\u * 1.5mm,\v * 1.5mm)  circle (0.02mm) ;  
     }
  }
  
   \draw [ |-| ] ( -15 * 1.5mm, -9 * 1.5mm) -- node[right] { $2 \pi / L$} ++ (0, 1.5mm);
  
  
\draw[->] (-2.5,0) -- (2.5,0) node [below] {$k_x$};  
  \draw[->] (0,-1.5) -- (0,1.5) node [right] {$k_y$};  

\draw[thick] (0,0) ellipse (1.9cm and 0.9cm);
\draw[] (0,0) ellipse (2.1cm and 1.2cm);

\node at (-10mm, 5mm) {$S(\omega)$};
\node at (-15mm, 12.5mm) {$S(\omega + d\omega)$};


\draw[->] (10mm,8mm) -- node (a) {} ++ (73:2mm);
\draw[->] (10mm,8mm) -- ++ (-17:2mm);

\node at (11.5mm, 12mm) {$dk_\perp$};
\node at (10mm, 5.5mm) {$dS_\omega$};

%  \tikzmath{\kx = 2.26; \ky = -1.8;}
%  \tikzmath{\x2 = \kx * \kx; \y2 = \ky * \ky;}
%  
%  \tikzmath{\kr = sqrt( \x2 + \y2 );}
%  
%  \coordinate (O) at (0,0);
%  \coordinate (A) at (\kx,\ky);
%  \coordinate (B) at (3,1);
%  
%  
% 
% \draw[<-,thick, shorten <=0.5mm] (O) -- node[right] {$\mathbf{k}_0$} (A); 
% 
% \draw[thick, gray]  (A) circle (\kr);
% 
% \draw[->,thick, shorten >=0.5mm] (O) -- node[above] {$\mathbf{G}$} (B); 
% \draw[->,thick, shorten >=0.5mm]  (A) -- node[right] {$\mathbf{k}$} (B); 
% 
%  \node[left] at (O) {(000)};
%  \node[above] at (B) {(310)};
%  
%  \pic [draw=black,<->,angle eccentricity=1.2,angle radius=0.5cm]
%           {angle=B--A--O };
%
%\node[xshift=-2mm, yshift=6mm] at (A) {$2 \Theta$};

  
  \end{tikzpicture}

\end{document}


