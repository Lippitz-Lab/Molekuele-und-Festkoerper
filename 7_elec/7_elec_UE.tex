\renewcommand{\lastmod}{\today}

\chapter{Elektronische Spektroskopie}



\section{Franck-Condon-Prinzip}

% ML14  auch AK17
 
Es sollen die vibronischen Ubergänge eines Moleküls vom elektronischen Grundzustand S0 in
einen Schwingungszustand des ersten angeregten elektronischen Niveaus S1 betrachtet werden. Die Schwingungszustännde werden in harmonischer Näherung beschrieben. Der S1 leftarriow S0 (1 leftarrow 0)
 
Übergang liege bei 15680cm-1 (welche Wellenlänge und Farbe ist das?) und der Abstand zweier

(b) Berechnen Sie die spektrale Position der Ubergänge S1 leftarrwo S0(n leftarrow 0), n = 1, 2, 3....
 
(c) Wodurch wird die Wahrscheinlichkeit für einen vibronischen Übergang bestimmt?
Die Eigenfunktionen der Schrödingergleichung für den harmonischen Oszillator lauten (bis auf einen Vorfaktor):
%􏰂 x2􏰃
%ψn(x) = exp − Hn s
sind die Hermiteschen Polynome.

(d) Geben Sie die Überlappintegrale für die Übergänge S1 leftarrow S0(n leftarrow 0), n = 1, 2, 3..., m an.

(e) Man erhält für die relative Intensität I(n) dieser Übergänge die folgende Poissonverteilung:
Sn n!
Berechnen Sie für S = 4.5 die relativen Intensitäten der ersten 8 Linien im Spektrum und skizzieren Sie dieses mit Hilfe der Ergebnisse aus Teilaufgabe (b). Welche Bedeutung könnte der Parameter S haben?



\section{Franck-Condon-Prinzip I}
% AK17

a) Erklären Sie kurz die Begriffe Born-Oppenheimer-Näherung und Franck-Condon-Prinzip.
b) Wodurch wird die Wahrscheinlichkeit für einen vibronischen Übergang (z.B. S1 ← S0 (m ← 0))
bestimmt?
Die Schwingungsfrequenz des Kohlenmonooxids CO im Grundzustand beobachtet man im Infrarot-


Absorprtionsspektrum bei Zimmertemperatur bei 155 nm und weist eine Schwigungsprogression mit einem Bandenabstand von 1480cm-1auf. Jede Schwingungsbande besteht aus rot-abschattierten P-, Q- und R-Zweigen, aus deren Analyse an die Rotationskonstanten B prime = 1,61cm-1 im angeregten und B doubleprime = 1,93cm-1 im Grundzustand gewinnt. Verwenden Sie zur Behandlung der Molekül-Vibration die harmonische Näherung mit dem parabolischen Potential:
V (r) = k2 (R - Re)2 (1) Gehen Sie bei der Rotation zudem von einem starren Rotator aus.

c) Berechnen Sie aus den gegebenen Daten die mittleren Kernabstände R doublepirme und R prime  für den elektro- ee
nischen Grundzustand und den angeregten Zustand, sowie die beiden jeweiligen Kraftkonstanten k doublepirme und k prime der molekularen Bindung.

d) Skizzieren Sie mit den Ergebnissen aus Teilaufgabe c) ein vereinfachtes Schema für die Potentialkurven der beiden Zustände. Zeichnen Sie dazu die Graphen der beiden harmonischen Potentiale maßstabsgetreu in ein Diagramm. Zeichnen sie auch die ersten drei Vibrationszustände des elektronisch angeregten Zustandes ein.

e) Ermitteln Sie nun anhand der Skizze aus Teilaufgabe d) mit Hilfe des Franck-Condon-Prinzips die stärksten vibronischen Übergänge des Spektrums.


\section{Lumineszenz}
%ML14


\begin{itemize}

\item[\textbf{(a)}] Welchen physikalischen Vorgang beschreibt der Begriff Lumineszenz? Inwiefern unterscheidet sich Lumineszenz von Streuung von Licht an Materie?
\item[\textbf{(b)}] Denken Sie sich ein einfaches Experiment aus, mit dem man das Fluoreszenzspektrum eines Farbstoffs messen könnte (schematisch).  
\item[\textbf{(c)}] Ein Molekül fluoreziert bei 400~nm mit einer Halbwertszeit von 1~ns durch den Übergang $S^* \rightarrow S$. Es phosphoresziert bei 500~nm durch den Übergang $T \rightarrow S$. Wenn das Verhältnis der Übergangswahrscheinlichkeiten für stimulierte Emission der beiden Übergänge $\frac{P(S^* \rightarrow S)}{P(T \rightarrow S)}~=~10^5$ beträgt, wie lange ist dann die Halbwertzeit der Phosphoreszenz?

\textit{Hinweis: Informieren Sie sich über die sogenannten Einstein-Koeffizienten für stimulierte und spontane Emission. Inwiefern besteht ein Zusammenhang zwischen den Einstein-Koeffizienten und Fluoreszenz- bzw. Phosphoreszenzhalbwertzeiten?}
 
\item[\textbf{(d)}] Sie möchten die Farbe eines Polyen-Farbstoffs verändern. Würden Sie das Polyen dazu verlängern oder verkürzen und begründen Sie Ihre Wahl? Würde sich die Farbe ins rötliche oder bläuliche verschieben? 
	
\end{itemize}



\section{Quiz zur Molekülphysik}
%ML^14

Prüfen Sie, ob folgende Aussagen korrekt sind und begründen Sie Ihre Antwort. \\
\textit{Hinweis:} Jede Teilaufgabe gibt einen halben Punkt.

\vspace{0.2cm}

\begin{itemize}

\item[\textbf{(a)}] O$_2^+$: $(1\sigma_g)^2(1\sigma_u)^2(2\sigma_g)^2(2\sigma_u)^2(3\sigma_g)^2(1\pi_u)^2(1\pi_u)^2(1\pi_g)^1$ \\ Aus der gegebenen Elektronenkonfiguration folgt das Termsymbol $^1\Pi_u$. 

\item[\textbf{(b)}] Bei der Hybridisierung werden die Wellenfunktionen von zwei Energie-entarteten Zuständen gemischt.

\item[\textbf{(c)}] Die J-Werte eines Übergangs im Rotationsspektrum können bestimmt werden durch die Aufspaltung der Linien im elektrischen Feld.

\item[\textbf{(d)}] Bei Änderung des Schwingungszustand eines Moleküls ändert sich immer auch der Rotationszustand.

\item[\textbf{(e)}] Wenn die Normalschwingungen von Benzol in der Gasphase untersucht werden sollen, müssen sowohl IR- als auch Raman-Spektren gemessen werden.

\item[\textbf{(f)}]  Aus dem Intensitätswechsel im Rotations-Ramanspektrum eines homonuklearen zweiatomigen Moleküls kann die Kernspin-Quantenzahl bestimmt werden.

\item[\textbf{(g)}] Bei elektronischen Anregungen beeinflusst die Änderung des Kernabstands im Molekül nur die Intensitätsverhältnisse der verschiedenen vibronischen Banden.
 
\item[\textbf{(h)}] Das Fluoreszenzspektrum eines Moleküls ist das exakte Spiegelbild des Absorptionsspektrums.
	
\end{itemize}
