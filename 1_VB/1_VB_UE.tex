\renewcommand{\lastmod}{\today}

\chapter{Die Valenzbindungstheorie und die Form von Molekülen}






\section{Aufwärmübung}

% ML 2014

In einem Gedankenexperiment werden die Wassermoleküle in 1 Liter Wasser markiert und dann ins Meer geschüttet, so dass sie sich gleichmäßig über alle Weltmeere verteilen. Schätzen Sie ab, wie viele markierte Wassermoleküle sich in 1 Liter Wasser befinden, den man wieder aus dem Meer entnimmt.
	


\section{ Wasserstoffatom}

% ML 2014

Die Lösung der Schödingergleichung für ein Wasserstoff-Atom ist eine bekannte Aufgabe aus der Quantenmechanik. Die Wellenfunktion des Elektrons $\Psi_{nlm}(r,\theta,\varphi)$ kann in einen Radialteil $R_{nl}(r)$ und einen winkelabhängigen Teil $Y_{lm}(\theta,\varphi)$ aufgespalten werden, mit den Quantenzahlen $n$, $l$ und $m$. \\

Die radialen Komponenten sind gegeben durch:

\begin{eqnarray*}
	& & R_{10}(r)=2 \left( \frac{Z}{a_0} \right) ^{3/2} \; e^ {-Zr/a_0}, \\
	& & R_{20}(r)=2 \left( \frac{Z}{2a_0}\right) ^{3/2}  \; \left( 1- { \frac{Zr}{2a_0} } \right) \; e^ {-Zr/(2a_0)}, \\
	& & R_{21}(r)= \frac{1}{\sqrt{3}}  \left( \frac{Z}{2a_0} \right) ^ {3/2} \;  {\frac{Zr}{a_0}} \; e^ {-Zr /(2a_0)}, \\
\end{eqnarray*}

wobei $a_0$ der Bohrsche Radius und $Z$ die Kernladungszahl ist. Die winkelabhängigen Komponenten $Y_{lm}(\theta,\varphi)$ sind die sphärisch harmonischen Kugelflächenfunktionen.

\vspace{0.2cm}

\begin{itemize}

	\item[\textbf{(a)}] Wir wollen nun untersuchen, wie sich die Wellenfunktion beim Annähern an das Atom verhält. Benutzen Sie ein Simulationsprogramm (z.B. Matlab, Maple, Mathematica, ...) um den Radialteil der Wellenfunktionen $\Psi_{100}, \Psi_{200}$ und $\Psi_{210}$ zu plotten. Was unterscheidet die Wellenfunktionen mit $l=0$ von denen mit $l=1$?

	\item[\textbf{(b)}] Das Quadrat des Absolutwerts der Wellenfunktion kann als die Wahrscheinlichkeitsdichte $w_{nlm}$ des Elektrons interpretiert werden:
\begin{equation*}
	w_{nlm}(r, \theta, \varphi) dV =\left|\Psi_{nlm}(r, \theta, \varphi) \right|^2 dV.
\end{equation*}

Betrachten Sie nun wieder den Radialteil $w_{100}^{\text{radial}}$ der Wellenfunktion $\Psi_{100}$. 
Berechnen Sie den wahrscheinlichsten Abstand des Elektrons vom Kern, indem Sie das Maximum der Wahrscheinlichkeitsdichte in diesem Zustand mithilfe der folgenden Beziehung bestimmen.

\begin{equation*}
w_{nlm}^{\text{radial}} (r) dr = \int_0^{\pi}\int_0^{2\pi}\Psi_{nlm}^{*} \Psi_{nlm} r^2~dr \sin \theta ~ d\theta~ d\varphi
\end{equation*}
	


	\item[\textbf{(c)}] Der mittlere Radius des Elektrons kann über den Erwartungswert bestimmt werden:

\begin{equation*}
\left\langle r \right\rangle = \left\langle \Psi_{nlm} \left| r \right| \Psi_{nlm} \right\rangle.
\end{equation*}

Betrachten Sie erneut die Wellenfunktion $\Psi_{100}$ und berechnen Sie den Erwartungswert von $r$ in diesem Zustand. Vergleichen Sie das Ergebnis mit dem Maximum der Wahrscheinlichkeitsdichte (siehe Teilaufgabe b).	
	
	\item[\textbf{(d)}] Nachdem wir uns mit dem einfachsten Orbital des Wasserstoffatoms beschäftigt haben, wollen wir uns noch den Orbitalen höherer Ordnung zuwenden. Nutzen Sie dazu zum Beispiel das Java-Applet "`Hydrogen Atom Orbital Viewer"' (http://www.falstad.com/qmatom/). Visualisieren Sie alle möglichen Fälle für $n=3$. Welche Symmetrien erkennen Sie?
	
\end{itemize}
	




\section{Hybridisierung}

% ML 2014

In dieser Aufgabe betrachten wir Atomorbitale und wie diese hybridisieren. Insbesondere wird die Frage beantwortet, warum nur manche Linearkombinationen der Orbitale neue hybridisierte Orbitale bilden. \\

In Graphit bilden die Kohlenstoffatome drei identische, d.h. energetisch entartete, Bindungen zu ihren Nachbaratomen aus. Die $2s$ und $2p$ Orbitale hybridisieren zu neuen Orbitalen $\psi_i$, welche Linearkombinationen der alten Orbitale $s$, $p_x$ und $p_y$ sind. Das $p_z$ Orbital bleibt unverändert. Für die alten und neuen Orbitale gilt:

\begin{eqnarray*}
&&\phi_s = e^{-r} (1-r), \\
&&\phi_{pz} = e^{-r} r \cos \theta, \\
&&\phi_{px} = e^{-r} r \sin \theta \cos \varphi, \\
&&\phi_{py} = e^{-r} r \sin \theta \sin \varphi, \\
&&\psi_i = a_{s,i} \phi_s + a_{px,i} \phi_{px} + a_{py,i} \phi_{py}.
\end{eqnarray*}

Orbitale sind immer orthonormierte Systeme, d.h. es gilt $\left\langle\phi_k | \phi_l\right\rangle = \delta_{kl}$, wobei $\delta_{kl} =1$ für $k=l$ und $\delta_{kl} =0$ sonst. Diese Orthonormierung gilt für die alten Orbitale $\phi$ und ist auch für die neuen Orbitale $\psi$ erforderlich. 

\vspace{0.2cm}

\begin{itemize}

	\item[\textbf{(a)}] Berechnen Sie die Koeffizienten $a$, so dass die $\psi_i$ orthonormiert sind. \\ \textit{Hinweis:} Sie haben 9 Unbekannte $a$ und nur 6 Gleichungen, benötigen also noch 3 zusätzliche Bedingungen. Zwei davon sind $a_{s,1}=a_{s,2}=a_{s,3}$. Warum muss dies gelten? Als letzte Bedingung wählen wir $a_{py,3}=0$. Was beeinflusst man mit dieser Wahl?

  \item[\textbf{(b)}] Plotten Sie mithilfe eines Computerprogramms die alten und die neuen Orbitale $\phi$ und $\psi$ um eine Vorstellung von der räumlichen Verteilung der Wellenfunktion zu bekommen. Kennzeichnen Sie die Gebiete mit positivem und negativem Vorzeichen. \\
	
\end{itemize}



\section{Valenzbindungstheorie (VB-Theorie)}

% AK17


a) Geben Sie die VB-Beschreibung von NH3 an und sagen Sie aufgrund diser Beschreibung den HNH-Winkel vorher. Vergleichen Sie die Vorhersage mit der Realität.
b) Verwenden Sie die VB-Theorie, um die Form des Wasserstoffperoxidmoleküls H2O2 vorherzusagen.

\section{Hybridisierung}
% AK17

Vor allem im Fall organischer Moleküle werden Bindungen häufig durch Hybridisierung beschrieben. Zu diesem Zweck werden alle entarteten Orbitale von Valenzelektronen der an der Bindung beteiligten Atome linear kombiniert (LCAO). Im folgenden gehe man von wasserstoffähnlichen Wellenfunk- tionen %Ψ 
zur Hauptquantenzahl n = 2, also 2s- und 2p-Zuständen aus.
a) Bestimmen Sie die Punkte maximaler (Ladungs-)Dichte bei diagonal oder sp-hybridisierten Elektronen. Gehen Sie von Wellenfunktionen der Form
%Ψ1/2≡Ψ±=√1 (Ψ2s±Ψ2π) (12) 2
%aus, mit Ψ2s ∝ (1 − γr) exp (−γr) sowie Ψ2π ∝ γxi exp (−γr), (xi = x, y, z) 
und maximieren Sie die Aufenthaltswahrscheinlichkeitsdichte als Funktion des Ortes.
b) Vergleichen Sie das Ergebnis aus a) mit den jeweiligen Ladungsschwerpunkten. Um die Schwerpunkte bzw. die Ortsmittelwerte zu bestimmen, muss die Wellenfunktion inklusive der Normierungsaktoren integriert werden:
%   1􏰂r􏰃􏰄r􏰅
%Ψ2s=√ 1− exp − 8πα3 2a
%2α
%(13) (14)
% 1x􏰄r􏰅
%Ψ2px=√ exp − 8πα3 2α
%2α
% 


\section{ Spinwellenfunktionen}
%AK17

Unter der Voraussetzung, dass die Spinwellenfunktion eines nach oben gerichteten Spins des Teilchens j mit $\alpha(j)$ und die eines nach unten gerichteten mit $\beta(j)$ bezeichnet werden, zeige man, dass die Wellenfunktionen
%√1 [α(1)β(2)+β(1)α(2)]Ψu und √1 [α(1)β(2)−β(1)α(2)]Ψg (11) 22
zum Gesamtspin 1 (mit verschwindender z-Komponente) beziehungsweise zum Gesamtspin 0 gehören. %Ψu/g
 bezeichne dabei die (un)gerade Superposition der Produkte zweier Einteilchenwellenfunktionen beim H2-Molekül.
Hinweis: Für die Komponenten des Gesamtspins gilt:
 Sz = S1z + S2z (x,y analog). Bestimmen Sie die Matrixelemente von S2 bzgl. der Basis
 % {|α(1)α(2)⟩,|α(1)β(2)⟩,|β(1)α(2)⟩,|β(1)β(2)⟩}. 
 Verwenden Sie dazu die Stufenoperatoren $S_\pm$ (vgl. Aufgabe 5).
 
 
\section{ Hybridorbitale}
% IR11
	- Animieren Sie Darstellungen der Orbitale des H-Atoms.
	
- Animieren Sie Hybridorbitale.


 