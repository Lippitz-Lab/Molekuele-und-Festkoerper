%\renewcommand{\lastmod}{\today}
\renewcommand{\chapterauthors}{Markus Lippitz}
\renewcommand{\lastmod}{7. November 2021}

\chapter{Die Valenzbindungstheorie und die Form von Molekülen}




\section{Ziele}
 


\begin{itemize}
\item Sie können die Valenzbindungstheorie benutzen, um die Form von Molekülen vorherzusagen und zu erklären. Ein Beispiel ist das hier abgebildete Pentacen-Molekül.

\item Sie können die Grundzüge verschiedener Methoden erklären, mit denen Eigenschaften von Molekülen bestimmt werden können.

\item Sie können die Begriffe Orbital, $\sigma$- oder $\pi$-Bindung und Hybridisierung erklären und korrekt verwenden.

\end{itemize}


\begin{marginfigure}
\includegraphics[width=\textwidth]{\currfiledir Pentacene_on_Ni(111)_STM.jpg}
\caption{Pentacen-Moleküle durch ein Raster-Tunnel-Mikroskop abgebildet.  Bildgröße $5 \times 5$~nm.
\href{https://commons.wikimedia.org/wiki/File:Pentacene_on_Ni(111)_STM.jpg}{L.~E.~Dinca et al.}  / \href{https://creativecommons.org/licenses/by/4.0}{CC BY}
}
\end{marginfigure}


\section{Überblick}


Die Valenzbindungstheorie (engl. valence bond theory) ist die historisch erste quantenmechanische Theorie zur Molekülbindung. Sie wurde 1927 von
Walter Heitler und Fritz London entwickelt. Kurz darauf entstand die komplementäre Molekülorbitaltheorie, die wir im nächsten Kapitel besprechen werden. Beide Theorien versuchen durch verschiedene Annahmen das quantenmechanische Problem zu lösen, welche Grundzustands-Energie und räumliche Anordnung die vielen Atomkerne und noch viel mehr Elektronen in einem Molekül annehmen.

Die Valenzbindungstheorie (VB) macht die Annahme, dass ein Elektron zunächst einmal zu einem einzigen Atomkern gehört und somit durch ein Atomorbital beschrieben wird. Eine Bindung zwischen Atomen entsteht durch das Paaren von Elektronen. Dies wird im Lewis-Schema dargestellt. In diesem Themenbereich sind oft Bücher aus der Chemie hilfreich, beispielsweise das Kapitel zur Molekülstruktur in \cite{Atkins}.
Die Molekülorbitaltheorie (MO) hingegen bildet die Gesamtwellenfunktion aufbauend auf der Annahme, dass jedes Elektron über das gesamte Molekül verteilt ist, also durch Molekülorbitale beschrieben wird. Die Valenzbindungstheorie ist einfacher, insbesondere von Hand, ohne Computer, und führt Begriffe ein, die auch darüber hinaus verwendet werden. Die Molekülorbitaltheorie ist insbesondere durch die Verwendung von Computern weiter entwickelt und 'moderner'.



\section{Moleküle 'sehen'}

\paragraph{Röntgenstreuung} Wenn sich viele identische Moleküle zu einem Kristall anordnen lassen (was nicht trivial ist), dann kann man kurzwellige elektromagnetische Strahlung an diesem  Kristall-Gitter beugen. Aus dem Beugungsbild erhält man durch Fourier-Transformation und ein paar Annahmen die Elektronendichte-Verteilung im Kristall. Es sind die Elektronen, die mit EM-Strahlung wechselwirken. Die Elektronendichte-Verteilung wird oft in Form von 'Höhenlinien' dargestellt. Damit ist klar, dass es nicht eine exakte 'Größe' eines Moleküls gibt.

\begin{marginfigure}
\includegraphics[width=\textwidth]{\currfiledir scattering.png}
\caption{Beugung von Röntgenstrahlen liefert die Fourier-Transformierte der Elektronenverteilung.
}
\end{marginfigure}


\paragraph{Raster-Tunnel-Mikroskopie} (engl. scanning tunneling microscope, STM) Direkter, ohne Fourier-Transformation, kann man einzelne Moleküle auf einer leitenden Oberfläche in einem sehr guten Vakuum abbilden, in dem man eine sehr scharfe Metallspitze nahe an die Oberfläche bringt. Bei einem Abstand im Bereich weniger Angstrom fließt ein Tunnelstrom, wenn eine Spannung zwischen Spitze und Oberfläche angelegt wird. Eine Elektronik hält den Tunnelstrom durch Regelung des Abstands konstant währen die Spitze über die Oberfläche gerastert wird. Man erhält ein Höhen-Bild konstanten Tunnelstroms. Da Strom nur fließen kann, wenn sich die Wellenfunktion der Elektronen in der Spitze und in der Oberfläche wenigstens etwas überlappen, kann man so die Elektronen-Wellenfunktion von Molekülen auf der Oberfläche abbilden. Die beobachtete Form hängt auch von der gewählten Spannung ab.

\begin{marginfigure}
\includegraphics[width=\textwidth]{\currfiledir stm.png}
\caption{Prinzip STM}
\end{marginfigure}


\paragraph{Raster-Kraft-Mikroskopie} (engl. atomic force microscope, AFM) Alternativ zum Tunnelstrom kann man die Kraft zwischen einer Spitze und der Probe bestimmen und Oberflächen konstanter Kraft abbilden. Wie wir in späteren Kapiteln sehen werden, gibt es einen anziehenden Abstands-Bereich aufgrund der van-der-Waals Wechselwirkung, und einen abstoßenden Bereich aufgrund des Pauli-Verbots. Als Sonde wird eine sehr scharfe Spitze aus beispielsweise Silizium verwendet, die an einer schwingenden Blattfeder montiert ist. Die Auslenkung dieses Arms wird gemessen und spiegelt die Wechselwirkung mit der Oberfläche wider.


\begin{marginfigure}
\includegraphics[width=\textwidth]{\currfiledir afm.png}
\caption{Prinzip AFM}
\end{marginfigure}

 
\begin{questions} 
\item Wie groß ist ein Molekül?
\item Welche physikalische Eigenschaft eine Moleküls wird bei Röntgenstreuung, STM und AFM abgebildet?
\end{questions}
 
 
\section{Vorbereitung}


\subsection{Wellenfunktionen des Wasserstoff-Atoms}

Wir benötigen an verschiedenen Stellen eine anschauliche Vorstellung der Wellenfunktionen eiens Elektrons im Wassertsoff-Atom, die von  der Quantenmechanik geliefetr werden. Welche Bedeutung haben die Quantenzahlen? Wei kann man eine Wellenfunktion grafisch darstellen? Was ist es eigentlich, was man da darstellt? Rekapitulieren Sie diese Kapitel der Atomphysik und Quantenmechanik noch einmal.



Das Pluto-Skript hydrogen\_wave\_functions\pluto{hydrogen_wave_functions.jl} ermöglicht es Ihnen, mit verschiedenen Varianten der grafischen Darstellung zu experimentieren.


\subsection{Orbital oder Wellenfunktion?}

Wir besprechen hier Systeme, die  aus vielen Elektronen bestehen. Die Quantenmechanik und Atomphysik konzentrierte sich jedoch auf das Wasserstoff-Atom mit nur einem Elektron. Wir müssen daher vorsichtig mit der Nomenklatur sein. Die (Gesamt-)Wellenfunktion eines Systems aus $n$ Elektronen ist im allgemeinen Fall $\Psi(\mathbf{r}_1, \mathbf{r}_2, \dots)$, wobei die $\mathbf{r}_i$ die Position des Elektrons $i$ bezeichnen. In dieser Allgemeinheit hängt alles miteinander zusammen und ist viel zu komplex. Wir machen daher immer die Annahme, dass sich die Gesamt-Wellenfunktion als Produkt von Orbitalen $\phi_i$ schreiben lässt
\begin{equation}
\Psi(\mathbf{r}_1, \mathbf{r}_2, \dots) = \phi_1(\mathbf{r}_1) \, \phi_2(\mathbf{r}_2)  \dots
\end{equation}
Die Orbitale hängen also nur von der Position 'ihres' Elektrons ab, nicht von all den anderen Elektronen. Im Fall des Wasserstoff-Atoms mit nur einem Elektron gehen die beiden Begriffe ineinander über.
Diese Aufteilung funktioniert immer, wenn die einzelnen Elektronen nicht miteinander wechselwirken, aber genau das ist der Fall. Diese Näherung versucht also, durch geschickte Wahl der $\phi_i$ diese Wechselwirkung vorweg zu nehmen. Es geht also darum, 'gute' $\phi_i$ zu finden.

\subsection{Born-Oppenheimer Näherung}

Atomkerne sind viel schwerer als Elektronen. In der Born-Oppenheimer Näherung betrachten wir die Kerne als stillstehend. Die Elektronen bewegen sich im stationären elektrischen Feld der Kerne. Diese Näherung wird quasi immer gemacht, so dass eigentlich nur erwähnt wird, wenn sie \emph{nicht} eingesetzt wird. Formal bedeutet dies, dass die Wellenfunktion des Moleküls geschrieben werden kann als Produkt der Wellenfunktion aller Elektronen und der Wellenfunktion aller Kerne, also
\begin{equation}
\Psi_{\text{Molekül}}(\mathbf{r}_1, \mathbf{r}_2, \dots, \mathbf{R}_1, \mathbf{R}_2, \dots)
  \approx
  \Psi_{\text{Elektronen}}(\mathbf{r}_1, \mathbf{r}_2, \dots )
\Psi_{\text{Kerne}}( \mathbf{R}_1, \mathbf{R}_2, \dots)
\end{equation}
wobei $\mathbf{r}_i$ Elektronenkoordinaten sind und $\mathbf{R}_i$ Kernkoordinaten.


%\begin{marginfigure}
%\includegraphics[width=0.8\textwidth]{\currfiledir potential.png}
%\caption{Die Energie-Eigenwert der Elektronen liefert das Bindungspotential.
%}
%\end{marginfigure}


Wir lösen also die Schrödingergleichung für freie Elektronen-Koordinaten, aber die Kern-Koordinaten werden als fix angenommen. Das \emph{Bindungspotential} stellt die Gesamtenergie des Systems dar, wenn für jeden Punkt der Kurve ein anderer aber jeweils fester Kern--Kern--Abstand angenommen wird. Eine Bindung kommt dann zustande, wenn das Bindungspotential ein Minimum hat. Der Kern--Kern--Abstand ist dann der Bindungsabstand.

\begin{questions} 
\item Was ist im Bindungspotential gebunden?
\end{questions}

\section{Das Wasserstoff-Molekül: $\sigma$-Bindung}


Wir betrachten zwei Wasserstoff-Atome $A,B$ mit insgesamt zwei Elektronen $1,2$. Der Hamilton-Operator für ein einzelnes Atom $\hat{H}_A$ ist\sidenote{$r_{ij} = |\mathbf{r}_i - \mathbf{r}_j|$}
\begin{equation}
  \hat{H}_A = - \frac{\hbar^2}{2 m_1} \nabla_1^2 - \frac{e^2}{4 \pi \epsilon_0} \frac{1}{r_{A1}}
\end{equation}
Die Wellenfunktion $\phi_A(\mathbf{r_1})$ löst damit die Schrödinger-Gleichung. Zwei sehr weit voneinander entfernte Wasserstoff-Atome haben die Gesamt-Wellenfunktion
\begin{equation}
 \Psi = \phi_A(\mathbf{r}_1) \, \phi_B(\mathbf{r}_2)
\end{equation}
%
\begin{marginfigure}
\includegraphics[width=\textwidth]{\currfiledir coords.png}
\caption{Skizze Koordinaten Atom und Molekül}
\end{marginfigure}
%
Wenn die beiden Atome nahe beieinander sind, kann man nicht mehr sagen, ob Elektron $1$ bei Kern $A$ oder Kern $B$ ist. In dieser Situation ist daher die Gesamt-Wellenfunktion 
\begin{equation}
 \Psi = \phi_A(\mathbf{r}_2) \, \phi_B(\mathbf{r}_1)
\end{equation}
genauso gut möglich. In der Quantenmechanik löst man dies auf durch die Superpostion der beiden Möglichkeiten. Die Gesamt-Wellenfunktion  ist daher 
\begin{equation}
 \Psi_{\pm} = \phi_A(\mathbf{r}_1) \, \phi_B(\mathbf{r}_2) \, \pm \, \phi_A(\mathbf{r}_2) \, \phi_B(\mathbf{r}_1) \label{eq:VB_psi_pm}
\end{equation}
Der Hamilton-Operator beinhaltet die beiden einzelnen Wasserstoff-Hamilton-Operatoren und insgesamt 4 Coulomb-Terme, die jeweils kreuzweise Ladungen aus den beiden Wasserstoff-Atomen verbinden
\begin{equation}
  \hat{H}_{ges} =  \hat{H}_A + \hat{H}_B  + \frac{e^2}{4 \pi \epsilon_0} \left( \frac{1}{r_{12}} + \frac{1}{r_{AB}} 
  - \frac{1}{r_{A2}}  - \frac{1}{r_{B1}} 
  \right) \quad. \label{eq:VB_Hges}
\end{equation}
%Um das Bindungspomtentail zu bestimmen kann der Term Term mit $r_{AB}$ auch zunächst weggelassen werden kann, weil die Kern-Positionen in der Born-Oppenheimer-Näherung ja als konstant angenommen werden.

Die Wellenfunktion Gl. \ref{eq:VB_psi_pm}  löst die Schrödinger-Gleichung mit dem Hamilton-Operator Gl. \ref{eq:VB_Hges} allerdings nicht. Man kann diese Wellenfunktion aber als Ausgangspunkt für das Variationsprinzip nehmen, das wir im nächsten Kapitel am Beispiel von \ch{H2+} besprechen werden. Es zeigt sich\footcite[Kapitel 4.4.2]{Haken_wolf_II}, dass die symmetrische Superposition die niedrigere Gesamtenergie liefert. Die beiden Terme der Wellenfunktion interferieren konstruktiv und sorgen dafür, dass im Raum zwischen den beiden Kernen eine höhere Aufenthaltswahrscheinlichkeit der Elektronen zu finden ist. Diese kompensiert die Coulomb-Abstoßung der Kerne.

In der Valenzbindungstheorie entstehen Bindungen zwischen Atomen durch das 
Paaren von zwei Elektronen. Zwei Kerne teilen sich also zwei Elektronen, die nicht mehr einem einzelnen Kern zugeordnet sind.

Der Spin der beteiligten Elektronen muss anti-symmetrisch gegen Vertauschung sein, da die Ortswellenfunktion ja symmetrisch ist, und das Pauli-Prinzip eine insgesamt anti-symmetrische Wellenfunktion  verlangt.

Die Ortswellenfunktion des Orbitals $\Psi_{+}$ ist wechselt nicht das Vorzeichen bei Rotation um die Kern--Kern--Achse. Dies ist analog zum s-Orbital im Wasserstoff-Atom bei Rotation um die (willkürlich gewählte oder durch das Magnetfeld gegebene) z-Achse.  Diese Bindung wird daher als \emph{$\sigma$-Bindung} bezeichnet. In beiden Fällen besitzt das Elektron keinen Drehimpuls.


\begin{marginfigure}
\inputtikz{\currfiledir gerade_ungerade}
\caption{Molekülorbitale, die hier aus atomaren 2s oder 2p-Orbitalen aufgebaut sind. Die Farbe kodiert das Vorzeichen der Wellenfunktion. Die Symmetrie $g$ oder $u$ ergibt sich aus der Punktspiegelung an der Mitte des Moleküls, hier durch den kleinen Punkt markiert. \label{fig:VB_AO_zu_MO}}
\end{marginfigure}


\begin{questions} 
\item Was bedeutet 'symmetrisch' für das Vorzeichen in Gl. \ref{eq:VB_psi_pm}? WIe sieht die dazu passende Spin-Wellenfunktion aus?
\end{questions}


\section{Die $\pi$-Bindung}

Die Art der Bindung hängt von der Orientierung der beteiligten Orbitale zueinander ab. Als Beispiel betrachte wir das Molekül \ch{N2}. Die Elektronenkonfiguration von Stickstoff ist [He]2s$^2$2p$_x^1$2p$_y^1$2p$_z^1$. Wir nehmen die z-Achse als Verbindungsachse der Kerne. Hier zeigen zwei p$_z$-Orbitale aufeinander. Eine $\sigma$-Bindung entsteht durch Paarung der beiden einzelnen Elektronen in den  p$_z$-Orbitalen, da die resultierende Wellenfunktion wieder rotationssymmetrisch um die Verbindungsachse sein wird. Sie hat die gleiche symmetrische Form wie oben.

Aus den p$_x$ und p$_y$-Orbitalen erhält man je eine $\pi$-Bindung: Die hantelförmigen p-Orbitale liegen parallel zueinander und senkrecht zur Bindungsachse. Dabei überlappen im Gegensatz zur $\sigma$-Bindung der p$_z$-Orbitalen beide Teile der Hanteln. Das Elektron in diesem bindenden Orbital hat einen Dreh\-impuls von $1 \hbar$ entlang der Kern--Kern--Achse, daher wird dies $\pi$-Bindung genannt.

%
\begin{marginfigure}
\includegraphics[width=\textwidth]{\currfiledir p-zu-sigma.png}
\caption{Atomare p-Orbitale können zu $\sigma$- und $\pi$-Bindungen kombinieren. }
\end{marginfigure}
%

Insgesamt ist \ch{N2} also aus einer $\sigma$-Bindung und zwei $\pi$-Bindungen aufgebaut.


\begin{questions} 
\item Decken sie die MO-Spalte in Abbildung  \ref{fig:VB_AO_zu_MO} ab und vergewissern Sie sich, dass Sie die Art der Bindung und die Bezeichnung der Orbitale angeben können.

\item Schauen Sie ggf. noch einmal in der Atomphysik nach, was eigentlich [He]2s$^2$2p$_x^1$2p$_y^1$2p$_z^1$ bedeutet und wo man diese Information findet.
\end{questions}



\section{Bindungswinkel  in  \ch{H2O}}


\textit{Lesen Sie Kapitel 9.7.1 Das \ch{H2O}-Molekül in \cite{Demtröder_ep3}. Wie kann man den Bindungswinkel von \ch{H2O} verstehen? In erster Näherung ergibt sich 90 Grad, in zweiter Näherung ein Wert, der näher am experimentell gefundenen liegt. \newline Schrieben und skizzieren Sie hier Ihre Erkenntnisse. }

\vspace*{5cm}

\newpage

\vspace*{5cm}

\begin{questions} 
\item Wie kommt es zur Form der Moleküle?
\end{questions}


\section{Hybridisierung von Kohlenstoff-Orbitalen}

Insbesondere in der organischen Chemie der Kohlenwasserstoffe spielt die Hybridisierung der Kohlenstoff-Orbitale eine wichtige Rolle. Im Kohlenstoff-Atom besteht nur ein geringer Energieunterschied zwischen der energetisch niedrigsten Elektronenkonfiguration
[He]2s$^2$2p$_x^1$2p$_y^1$2p$_z^0$ und der nächst höheren [He]2s$^1$2p$_x^1$2p$_y^1$2p$_z^1$. Dies bedeutet, dass der Energieunterschied zwischen dem 2s und dem 2p-Orbital in Kohlenstoff sehr gering ist, und insbesondere 
%
\begin{marginfigure}
\includegraphics[width=0.9\textwidth]{\currfiledir hybrid.png}
\caption{Elektronische Niveaus bei der Hybridisierung von Kohlenstoff. }
\end{marginfigure}
%
ist der Energiegewinn durch die Bindung sehr oft größer als dieser Unterschied. Es ist daher oft energetisch günstiger, die Bindung ausgehend von einer Linearkombination von 2s und 2p-Orbitalen zu betrachten. Dies nennt man Hybridisierung der Orbitale. Wenn ein s-Orbital und drei p-Orbitale beteiligt sind, dann wird dies als sp$^3$-Hybridisierung bezeichnet. Ohne Hybridisierung könnte Kohlenstoff nur zwei Bindungen eingehen (mit den p$_x$ und p$_y$-Orbitalen), nach sp$^3$-Hybridisierung vier, so dass die Gesamtenergie stärker abgesenkt werden kann.\sidenote{Auch ist die Idee eines s- oder p-Orbitals ein Ein-Elektron-Konzept, das in Mehrelektronen-Atomen durch die anderen Elektronen gestört wird.}

Die neuen Hybrid-Orbitale $h_{1 .. 4}$ sind so gewählt, dass $\braket{h_i | h_j} = \delta_{ij}$, also
\begin{eqnarray}
 h_1 = s + p_x + p_y + p_z \\
 h_2 = s + p_x - p_y - p_z \\
 h_3 = s - p_x + p_y - p_z \\
 h_4 = s - p_x - p_y + p_z 
\end{eqnarray}
Diese Orbitale entstehen also durch Interferenz der ursprünglichen Orbitale und haben eine Ladungsverteilung, deren Keulen einen Tetraeder aufspannen. Der Bindungswinkel ist $\arccos (-\frac{1}{3}) = 109.5^\circ$. Methan (\ch{CH4}) ist daher tetraederförmig.
%
\begin{marginfigure}
\includegraphics[width=\textwidth]{\currfiledir ch4.png}
\caption{sp$^3$-Hybridisierung in \ch{CH4}. }
\end{marginfigure}
%


Analog gibt es auch die sp$^2$ und die sp-Hybridisierung.  Die sp$^2$-Hybridisierung findet man beispielsweise in Ethen (\ch{C2H4}). Die drei  sp$^2$-Orbitale jedes Kohlenstoff-Atoms sind an der $\sigma$-Bindung der beiden Wasserstoff-Atome beteiligt und $\sigma$-Bindung zwischen den beiden Kohlenstoff-Atomen. Die zweite C--C Bindung ist eine 'gewöhnliche' $\pi$-Bindung zwischen den verbleibenden, nicht hybridisierten p-Orbitalen, die senkrecht auf die durch die sp$^2$-Orbitale gebildete Ebene stehen. Dadurch ergeben sich die Winkel in der HCH bzw. HCC-Bindung zu circa 120$^\circ$.
%
\begin{marginfigure}
\includegraphics[width=\textwidth]{\currfiledir c2h4.png}
\caption{sp$^2$-Hybridisierung in C$_2$H$_4$. }
\end{marginfigure}
%

Ein Beispiel für die sp-Hybridisierung ist Ethin (\ch{C2H2}, \ch{HC+CH} ).

\begin{questions} 
\item Wie entscheidet sich, ob die 'normalen' oder die 'hybriden' Orbitale zum Einsatz kommen?
\item Was ist bei der  Hybridisierung  so besonders an Kohlenstoff?
\end{questions}


\section{Zusammenfassung}

\textit{Schreiben Sie hier ihre persönliche Zusammenfassung des Kapitels auf. Konzentrieren Sie sich auf die wichtigsten Aspekte und die am Anfang genannten Ziele des Kapitels.}

\vspace*{10cm}

\printbibliography[segment=\therefsegment,heading=subbibliography]
